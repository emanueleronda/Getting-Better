\chapter{Concetti Introduttivi}

    \section{Anelli, Moduli e Campi}
        Un anello è una terna ordinata $(R,+,\cdot)$, tale che $R$ è un insieme non vuoto, $(R,+)$ è un gruppo abeliano, la moltiplicazione 
        è associativa su $R$ e valgono le seguenti leggi distributive: $a(b + c) = ab + ac, (b + c)a = ba + ca$ per ogni $a,b,c \in R$. Se 
        anche la moltiplicazione è commutativa diremo che l'anello è commutativo. Infine se esiste un elemento $e \in R$ tale che per ogni 
        $a \in R$, vale che $ae = ea = a$, tale elemento è detto identità, è unico e l'anello è detto con identità.
        \begin{comment}
            Un sottoinsieme $I \subseteq R$ si dice ideale se: è non vuoto, è un sottogruppo per la somma e per ogni $a \in I, r \in R, ar \in I$  
        e si indica con $I \ideal R$. Sia $a \in R$, si indica con $(a)$ l'ideale principale generato da $a$, ed è definito come $(a) = \{ar : 
        r \in R\}$. Si dice che gli elementi $r_1,\ldots,r_n \in R$ generano l'ideale $I$ e si scrive $I = (r_1,\ldots,r_n)$ se per ogni $a \in 
        I$ esistono $a_1,\ldots,a_n \in R$ tali che $a = \sum_{i=1}^n a_ir_i$.
        \end{comment}
        \begin{definizione}
            Un anello $R$ in cui ogni ideale è finitamente generato è detto \emph{noetheriano}.
        \end{definizione}
        D'ora in poi verranno considearti solo anelli commutativi con identità e, con abuso di nomenclatura, mi riferirò a questi come anelli.
        \begin{esercizio}\label{ex:noet-ring}
            Sia $R$ un anello noetheriano, e sia $\{r_i\}_{i \in \N} \subseteq R$ una successione di elementi di $R$ tali che $(r_i) \subseteq 
            (r_{i+1})$ per ogni $i$. Allora esiste $n \in \N$ tale che $(r_n) = (r_j)$ per ogni $j \geq n$.
        \end{esercizio}
        \begin{proof}
            Considero l'ideale $I = (r_i : i \in \N)$; siccome $R$ è noetheriano, esiste $n \in \N$ tale che $I = (r_0,\ldots,r_n)$. Affermo che $(r_n) = 
            (r_j)$ per ogni $j \geq n$. Sia dunque $j \geq n$ fissato. \\
            L'inclusione $(r_n) \subseteq (r_j)$ è data per ipotesi. Viceversa siccome $r_j \in I = (r_0,\ldots,r_n)$, esistono $a_0,\ldots,a_n \in R$ tali 
            che $r_j = \sum_{i=0}^n a_ir_i$, ma siccome $r_i \in (r_n) \, \forall i \leq n$, si ha che $r_j = \sum_{i=0}^n a_ir_i \in (r_n)$.
        \end{proof}
        \begin{lemma}\label{lem:loc-ring}
            Sia $R$ un anello. Le seguenti affermazioni sono equivalenti: 
            \begin{enumerate}[label = \alph*]
                \item L'insieme degli elementi non invertibili in $R$ ha la struttura di ideale
                \item $R$ ha un unico ideale massimale che contiene tutti gli altri ideali propri di $R$.
            \end{enumerate}
        \end{lemma}
        \begin{proof}
            Una dimostrazione di questo fatto si può trovare in \cite{fulton} Capitolo 2, Paragrafo 4.
            \begin{comment}
            Sia $M = \{r \in R : r \text{ non è un invertibile in } R\}$ un ideale di $R$. Allora, se $I \ideal R, I \neq R$, allora $I \subseteq
            M$, perché se esistesse $a \in I \setminus M$, allora $a$ sarebbe un invertibile di $R$, ma quindi $I = (a) = R$, contro l'ipotesi 
            che $I$ fosse ideale proprio. Quindi $M$ contiene tutti gli ideali propri di $R$ ed in particolare è massimale. \\
            Viceversa, sia $R$ che contiene un unico ideale massimale, che contiene tutti gli ideali propri di $R$ e sia $M$ tale ideale. Dimostro 
            che $$M = \{r \in R : r \text{ non è invertibile in } R\}$$ \\
            Sia $r \in M$, allora $r$ non è invertibile perché altrimenti $R = (r) \subseteq M$, ma $M$ è ideale proprio. Viceversa, se $r$ è un non 
            invertibile allora $I = (r)$ è un ideale proprio di $R$, quindi, $(r) \subseteq M \Longrightarrow r \in M$.
            \end{comment}
        \end{proof}
        Un anello che rispetta una (e quindi entrambe) delle condizioni del Lemma~\ref{lem:loc-ring} è detto \emph{anello locale}.\\
        \begin{lemma}\label{lem:DVR}
            Sia $R$ un dominio che non è un campo. Allora sono equivalenti le seguenti affermazioni: \begin{enumerate}[label = \alph*]
                \item $R$ è noetheriano, locale e tale che l'ideale massimale sia principale.
                \item $R$ è tale che esiste un elemento irriducibile $t \in R$ tale che per ogni altro elemento non nullo di $r \in R$, esistono unici 
                un invertibile $u \in R$ e $n \in \N$ tali che $r = ut^n$.
            \end{enumerate}
        \end{lemma}
        \begin{proof}
            Sia $R$ noetheriano, locale e con ideale massimale principale. Sia $M$ tale ideale e sia $t \in R$ un suo generatore.\\
            Dimostro che, per ogni $r \in R, r \neq 0$, esistono $u,n$ come nella seconda condizione: sia $r \in R, r \neq 0$ fissato. \\
            Se $r$ è un invertibile, basta scegliere $u = r, n = 0$ e si conclude. Sia quindi $r$ un non invertibile: allora, $r \in M$, ed esiste $r_0 \in R$ 
            tale che $r = r_0t$; se $r_0$ è un invertibile ho concluso, altrimenti $r_0 \in M$, ed esiste $r_1 \in R$ tale che $r_0 = r_1t$. Itero l'argomento. \\
            Affermo che dopo un numero finito di passi trovo un $r_i$ che è un invertibile. Se per assurdo così non fosse, costruisco una successione 
            $\{r_i\}_{i \in \N} \subseteq R$ di elementi non invertibili e non nulli tali che $(r_i) \subseteq (r_{i+1})$ per ogni $i$. Per l'Esercizio \ref{ex:noet-ring} 
            esiste un elemento massimale nella catena degli ideali principali, ovvero, esiste $n \in \N$ tale che $(r_n) = (r_j) \, \forall j \geq n$, in particolare 
            $(r_n) = (r_{n+1})$, ma allora, esiste $s \in R$ tale che $r_{n+1} = sr_n$, perciò: \begin{equation*}
                r_n = r_{n+1}t = sr_nt = str_n \Longrightarrow st = 1
            \end{equation*}
            Ma questo è assurdo perché $t$ non è invertibile.\\
            Dimostro l'unicità della scrittura: sia $r \in R$, e siano $u,v \in R$ invertibili ed $m,n \in \N$ tali che $r = ut^m = vt^n$. Ne segue che $ut^{m-n} = v$ 
            dunque $m = n$ e di conseguenza $u = v$. \\
            Viceversa, sia $R$ tale che esiste un elemento irriducibile $t$, tale che per ogni altro elemento $r \in R, r \neq 0$, esistono unici $u \in R$ invertibile 
            e $n \in \N$ tali che $r = ut^n$. \\
            Chiaramente $M = (t)$ è un ideale massimale, e se $r \in R$ non è invertibile, per ipotesi è in $M$; viceversa se in $M$ ci fosse un invertibile, allora 
            $M = R$, ma questo è assurdo perché $t$ è irriducibile. Ne segue che $M$ contiene tutti e soli i non invertibili. Questo dimostra che $R$ è locale.\\
            Inoltre, essendo $M$ l'unico ideale massimale, è principale perché generato da $t$. \\
            Sia ora $I \ideal R$ non banale e diverso da $M$. Essendo $R$ locale, $I \subseteq M$. Sia $r \in I$ non nullo, allora esiste $u \in R$ invertibile tale che $r = ut^n$ 
            per un opportuno naturale $n$. Sia $m = \min\{n \in \N : r = ut^n, r \in I, r \neq 0\}$. Dimostro che $I = (t^m)$.\\
            Sia $r \in I$, allora $r = ut^n$ per opportuni $u \in R$ invertibile, $n \in \N, n \geq m$, dunque $r = ut^{n-m}t^m \in (t^m)$.\\
            Viceversa, esiste $r \in I$ tale che $r = ut^m$, per un opportuno invertibile $u$, allora $t^m = u^{-1}r \in I$.
        \end{proof}
        Un anello che rispetta una (e quindi entrambe) delle condizioni del Lemma \ref{lem:DVR} è detto \emph{anello di valutazione discreta} e si scrive che è un DVR. 
        Un elemento $t \in R$ come nella seconda condizione è detto \emph{parametro uniformizzante}. Parametri uniformizzanti distinti sono tra loro associati. \\
        Sia ora $K$ il campo dei quozienti di $R$ e sia $t$ un parametro uniformizzante fissato: si osserva semplicemente che ogni elemento non nullo $z \in K$ ammette 
        un'unica scrittura nella forma $z = ut^n$, dove $u$ è un'invertibile in $R$ e $n \in \Z$. L'esponente $n$ è detto \emph{ordine} di $z$ e si scrive $n = \ord(z)$.
        Si pone $\ord(0) = \infty$. \\
        L'ordine di un elemento di $K$ è ben definito, ovvero, non dipende dalla scelta del parametro uniformizzante.
        \begin{proof}
            Siano $t,s \in R$ due parametri uniformizzanti e sia $u \in R$ invertibile tale che $t = us$. Sia ora $z \in K$ (con le stesse notazioni di sopra), e siano 
            $n_t,n_s$ gli ordini di $z$ calcolati a partire da $t$ e da $s$ rispettivamente. Allora, per un opportuno invertibile $v \in R$: \begin{equation*}
                z = vt^{n_t} = v(us)^{n_t} = vu^{n_t}s^{n_t}
            \end{equation*}
            e per l'unicità della scrittura con il parametro uniformizzante, $n_t = n_s$.
        \end{proof}
        \begin{osservazione}
            Vale che $R = \{z \in K : \ord(z) \geq 0\}$ e $M = \{z \in K : \ord(z) > 0\}$.
        \end{osservazione}
        \begin{definizione}
            Sia $R$ un anello ed $M$ un insieme non vuoto, allora $M$ si dice $R$\emph{-modulo} se $M$ è dotato di una operazione $+$ rispetto alla quale è un gruppo abeliano 
            ed esiste un'azione di $R$ su $M$, indicata come $\cdot: R \times M \to M$ tale che :
            \begin{itemize}
                \item $(a+b)m = am + bm,\,\forall a,b \in R, m \in M$;
                \item $a(m+n) = am + an, \, \forall a \in R, m,n \in M$;
                \item $(ab)m = a(bm), \, \forall a,b \in R, m \in M$;
                \item $1_R m = m, \, \forall m \in M$.
            \end{itemize}
        \end{definizione}
        Se $N \subseteq M$ è non vuoto, chiuso rispetto alla somma ed al prodotto per scalare, allora $N$ è detto \emph{sotto-}$R$\emph{-modulo} di $M$. Il sotto-$R$-modulo 
        generato da $S \subseteq M$ è l'insieme $M(S) = \{\sum_{i=0}^k r_is_i : r_i \in R, s_i \in S \, \forall i \leq k, k \in \N\}$.\\
        Sia ora $X$ un insieme qualsiasi e considero l'insieme $M_X = \{\gvf : X \to R \}$, con la somma definita puntualmente ed il prodotto per scalare definito anch'esso puntualmente. 
        Allora $M_X$ è un $R$-modulo, ed è detto $R$\emph{-modulo libero su} $X$. Sia ora $x \in X$ e sia $\gvf_x \in M_X$ definita come $\gvf_x(y) = 0$, se $x \neq y$ e $\gvf_x(x) = 1$, 
        allora $X \subseteq M_X$. \\
        Siano $K \leq L$ campi. Indico l'estensione di campi con $\frac{L}{K}$
        \begin{definizione}
            Un elemento $x \in L$ si dice \emph{algebrico} su $K$, se esiste un polinomio $F \in K[X]$, tale che $F(x) = 0$, \emph{trascendente} altrimenti. Allora $K[x]$ è il più piccolo 
            anello che contiene sia $K$ che $x$. Il suo campo dei quozienti è $K(x)$ ed è il più piccolo campo contenete sia $K$ che $x$.\\
            L'estensione $\frac{L}{K}$ si dice \emph{algerbica} se ogni $x \in L$ è algebrico su $K$.
        \end{definizione}
        Osservo ora che $L$ ha una struttura di spazio vettoriale su $K$; allora, si dice che l'estensione $\frac{L}{K}$ è \emph{finita} se $[L:K] = \dim_KL$ è finita.
        \begin{esercizio}\label{ex:fields}
            Siano $K \leq L$ campi e sia $L$ un modulo finitamente generato su $K$. Allora per ogni anello $K \leq R \leq L$, $R$ è un campo.
        \end{esercizio}
        \begin{proof}
            Sia $r \in R, r \neq 0$ un elemento algebrico su $K$, allora esiste un polinomio monico $F \in K[X]$ tale che $F(r) = 0$, sia $F(X) = \sum_{i=0}^n a_iX^i$. 
            Considero il polinomio $G(X) = \sum_{i=0}^n a_iX^{n-i}$; allora $G(r^{-1}) = 0$, moltiplicando per $r^{1-n}$ e riordinando si trova che $r^{-1}$ è combinazione 
            di elementi di $R$, dunque è in $R$. Se $r$ non è algerbico su $K$, il più piccolo modulo su $K$ che contiene $K[r]$ non è finitamente generato, ma $L$ contiene 
            tale modulo ed $L$ è finitamente generato per ipotesi. Dunque un tale elemento non può esistere. Ne segue che $R$ è un campo.
        \end{proof}
        \begin{teorema}[Dell'elemento primitivo]
            Sia $K$ un campo di caratteristica $0$, e sia $\frac{L}{K}$ un'estensione algerbica finita. Allora, esiste $\alpha \in L$, tale che $L = K(\alpha)$.
        \end{teorema}
        \begin{proof}
            Una versione più generale di questo risultato è dimostrata in \cite{milneFT} Capitolo 5.
        \end{proof}
    
        
    \newpage
    \section{Spazi affini e spazi proiettivi}\label{par:aff-proj}
        Sia $k$ un campo. Uno \emph{spazio affine di dimensione} $n$ \emph{su} $k$ è una terna ordinata $\A = (A,V,+)$, dove $A$ è un insieme, detto insieme dei punti, 
        $V$ è uno spazio vettoriale su $k$ di dimensione $n$, in biiezione insiemistica con $A$ e $+ : A \times V \to A$ è un'azione di $V$ su $A$ 
        tale che, $P + v = P \iff v = 0, \, \forall P \in A$ e $\forall P,Q \in A$ esiste $v \in V$ tale che $P + v = Q$. \\
        Se $A = V = k^n$, allora lo spazio $\A$ è detto \emph{spazio affine standard di dimensione} $n$ \emph{sul campo} $k$ e si denota con $\A^n(k)$, o anche con $\A^n$. 
        In tal caso, $P \in \A^n$, si indica con $P = (x_1,\ldots,x_n)$ e queste sono dette coordinate di $P$.\\
        Considero ora lo spazio vettoriale $k^{n+1}$. Definisco su $k^{n+1}$ la relazione $\sim$, 
        definita per ogni $x,y \in k^{n+1}$ da $x \sim y \iff$ esiste $\lambda \in k, \lambda \neq 0$ tale che 
        $x = \lambda y$. \\
        La relazione definita è un'equivalenza su $k^{n+1}$. \begin{comment}
        \begin{proof}
            $\sim$ è riflessiva: per la scelta $\lambda = 1$, segue che per ogni $x \in k^{n+1}$ si ha $x = \lambda x$. \\
            $\sim$ è simmetrica: per $x,y \in k^{n+1}$ tali che $x \sim y$, allora esiste $0 \neq \lambda \in k^{n+1}$ tale 
            che $x = \lambda y$, da cui $y = \lambda^{-1}x$, e di conseguenza la simmetria di $\sim$. \\
            $\sim$ è transitiva: se $x,y,z \in k^{n+1}$ sono tali che $x \sim y, y \sim z$, esistono $\lambda,\mu \in k$ 
            entrambi non nulli, tali che $x = \lambda y, y = \mu z$, per cui $x = \lambda\mu z$, perciò $x \sim z$.
        \end{proof} \end{comment}
        Sia dunque $\P^n(k)$, o semplicemente $\P^n$, l'insieme quoziente $\frac{k^{n+1}}{\sim}$. Questo insieme è detto \emph{
        spazio proiettivo} standard di dimensione $n$ sul campo $k$. I suoi elementi sono detti \emph{punti proiettivi}. Inoltre 
        se $x = (x_1,\ldots,x_{n+1}) \in k^{n+1}$ e $P \in \P^n$ è la sua immagine nel quoziente si scrive $P = [x_1,\ldots,x_{n+1}]$ 
        e si dice che $[x_1,\ldots,x_{n+1}]$ sono delle \emph{coordinate omogenee} del punto proiettivo $P$. \\
        Fissato un sistema di coordinate omogenee su $\P^n$, considero, per ogni $i \leq n$, l'insieme $U_i = \{P = [x_1,\ldots,x_{n+1}] \in \P^n : x_i \neq 0 \footnote{Questa 
        proprietà è indipendente dalla scelta delle coordinate omogenee.}\}$, e la mappa 
        $\varphi_i : U_i \to \A^n$, definita da $\varphi_i([x_1,\ldots,x_{n+1}]) = (\frac{x_1}{x_i},\ldots,\frac{x_{i-1}}{x_i},\frac{x_{i+1}}{x_i},\ldots,\frac{x_n}{x_i})$, 
        ciascuna di queste mappe è una biiezione.




    \newpage
    \section{Insiemi Algebrici}
        Sia d'ora in poi $k$ un campo algebricamente chiuso di caratteristica $0$, e siano $\A^n,\P^n$ lo spazio affine e lo spazio proiettivo standard di dimensione $n$ su $k$.
        \subsection{Caso Affine}
            \begin{definizione}
                Sia $S \subseteq k[X_1,\ldots,X_n]$, definisco \emph{l'insieme algebrico affine} $V(S) = \{P = (x_1,\ldots,x_n) \in \A^n : 
                F(P) = 0 \forall F \in S\}$. Se $I \ideal k[X_1,\ldots,X_n]$ è l'ideale generato da $S$, vale che $V(I) = V(S)$. Un insieme 
                algebrico affine $V$ è detto \emph{irriducibile} se non è unione di insiemi algebrici affini strettamente contenuti in $V$. 
                Un insieme algebrico affine irriducibile è detto \emph{varietà affine}.
            \end{definizione}
            \begin{proposizione}
                Unione finita di insiemi algebrici è un insieme algebrico. Intersezione arbitraria di insiemi algebrici è un insieme algebrico. $\emptyset,\P^n$ 
                sono insiemi algebrici.
            \end{proposizione}
            \begin{proof}
                La dimostrazione di questo fatto nel caso affine è analoga a quella del caso proiettivo nella proposizione \ref{prop:alg=clo}
            \end{proof}
            \begin{definizione}
                Sia $X \subseteq \A^n$, definisco l'ideale associato ad $X$ come $I(X) = \{F \in k[X_1,\ldots,X_n] : F(P) = 0 \, \forall P \in X\}$.
            \end{definizione}     
            \begin{osservazione}
                La definizione è ben posta, ovvero $I(X)$ è effettivamente un ideale per ogni $X$.
            \end{osservazione}
            \begin{proposizione}
                Un insieme algebrico $V$ è irriducibile se e solo se $I(V)$ è un ideale primo.
            \end{proposizione}
            \begin{proof}
                Sia $V$ irriducibile e siano $F,G \in k[X_1,\ldots,X_n]$ tali che $FG \in I(V)$. Allora, considero gli insiemi $V(F) = \{P \in V : F(P) = 0\}$ e $V(G) = \{
                P \in V : G(P) = 0\}$. Chiaramente $V(F) \cup V(G) \subseteq V$. Inoltre siccome $FG \in I(V), F(P)G(P) = 0$, quindi per ogni $P \in V, F(P) = 0$ oppure $G(P)
                = 0$, dunque $V \subseteq V(F) \cup V(G)$. \\
                Ma $V$ è irriducibile, quindi $V = V(F)$ oppure $V = V(G)$, da cui $F \in I(V)$ oppure $G \in I(V)$. \\
                Viceversa sia $V$ tale che $I(V)$ sia primo e siano $V_1,V_2 \subseteq V$ insiemi algebrici tali che $V = V_1 \cup V_2$. Se $V_1 = \emptyset$ oppure $V_2 = 
                \emptyset$, allora l'altro è uguale a $V$ e non c'è nulla da dimostrare. Suppongo quindi $V_1 \neq \emptyset \neq V_2$. Allora $I(V) \subseteq I(V_1),I(V_2)
                \Longrightarrow I(V) \subseteq I(V_1) \cap I(V_2)$. Viceversa: \begin{equation*}
                    F \in I(V_1) \cap I(V_2) \Longrightarrow F(P) = 0 \forall P \in V_1 \cup V_2 = V \Longrightarrow F \in I(V)
                \end{equation*}
                Vale che $I(V) = I(V_1) \cap I(V_2)$. Sia $F \in I(V_1) \setminus I(V)$, allora, per ogni $G \in I(V_2)$, essendo $FG \in I(V)$ ed $F \notin I(V)$, allora, 
                $G \in I(V)$, da cui $V_2 = V$.  
            \end{proof}
            Sia dunque $V \subseteq \A^n$ un insieme algebrico irriducibile e sia $I(V)$ il suo ideale primo associato. Allora considero l'anello $\gG(V) = 
            \frac{k[X_1,\ldots,X_n]}{I(V)}$. Siccome $I(V)$ è primo, $\gG(V)$ è un dominio ed è detto \emph{anello coordinato associato} a $V$.\\
            Siccome $\gG(V)$ è dominio, allora è ben definito il suo campo dei quozienti. Sia $k(V)$. Tale campo è detto \emph{campo delle funzioni razionali su} $V$.
            Siano ora $P \in V, z \in k(V)$ fissati; si dice che $z$ è definita in $P$, se esistono $f,g \in \gG(V)$, tali che $z = \frac{f}{g}$ e $g(P) 
            \neq 0$. Si definisce a questo punto $\cO_P(V) = \{z \in k(V) : z \text{ è definita in } P\}$. $\cO_P(V)$ è un anello locale, con ideale massimale $M_P(V) = \{
            z \in \cO_P(V) : z(P) = 0\}$.
        \subsection{Caso Proiettivo}
            \begin{definizione}
                Un punto $P \in \P^n$ si dice \emph{zero} del polinomio $\, F \in k[X_1,\ldots,X_{n+1}]$ se per ogni scelta $[x_1,\ldots,x_{n+1}]$ 
                di coordinate omogenee per $P$, vale che $F(x_1,\ldots,x_{n+1}) = 0$, e si scrive $F(P) = 0$.
            \end{definizione}
            Vale il seguente:
            \begin{lemma}
                Sia $F \in k[X_1,\ldots,X_{n+1}]$ un polinomio di grado $d$, e siano $F_0,\ldots,F_d \in k[X_1,\ldots,X_{n+1}]$ polinomi omogenei tali che 
                $F = \sum_{i=0}^d F_i$ e $F_i$ ha grado $i$. Allora un punto $P \in \P^n$ è zero di $F$ se e solo se è zero di $F_i$ per ogni $i$.
            \end{lemma}
            \begin{proof}
                Suppongo che $P$ è uno zero di $F$, allora, se $[x_1,\ldots,x_{n+1}]$ sono coordinate omogenee per $P$, per ogni 
                $\lambda \in k, \lambda \neq 0$, anche $[\lambda x_1, \ldots, \lambda x_{n+1}]$ sono coordinate omogenee per $P$. Dunque:
                \begin{multline*}
                    0 = F(x_1,\ldots,x_{n+1}) = F(\lambda x_1,\ldots,\lambda x_{n+1}) = \\
                    = \sum_{i=0}^d F_i(\lambda x_1,\ldots,\lambda x_{n+1}) 
                    = \sum_{i=0}^d \lambda^i F_i(x_1,\ldots,x_{n+1}) = G(\lambda)
                \end{multline*}
                Osservo ora che l'equazione polinomiale $G(t) = 0$ ha un numero infinito di soluzioni, dunque il polinomio $G$ è nullo, ovvero 
                $F_i(x_1,\ldots,x_{n+1}) = 0$ per ogni scelta di coordinate omogenee $[x_1,\ldots,x_{n+1}]$ per $P$ per ogni $i$, da cui la tesi. \\
                Viceversa, se $P$ è zero di $F_i$ per ogni $i$, per ogni scelta di coordinate proiettive $[x_1,\ldots,x_{n+1}]$ per $P$:
                \begin{equation*}
                    F(x_1,\ldots,x_{n+1}) = \sum_{i=0}^d F_i(x_1,\ldots,x_{n+1}) = 0
                \end{equation*}
            \end{proof}
            Sia ora $S \subseteq k[X_1,\ldots,X_{n+1}]$, allora definisco $V(S) = \{P \in \P^n : F(P) = 0 \forall F \in S\}$. Chiaramente se $I$ è l'ideale 
            generato da $S$, vale che: $V(I) = V(S)$. Un tale insieme è detto \emph{insieme algebrico proiettivo}.\\
            Osservo ora che siccome $k[X_1,\ldots,X_{n+1}]$ è noetheriano, $I$ è finitamente generato, ovvero $I = (F^1,\ldots,F^r)$. Ciascuno degli $(F^i)_{i=1}^r$ 
            può essere scritto come somma di polinomi omogenei nella forma $F^i = \sum_{j=0}^{d_i}F_j^i$, con $d_i$ grado di $F_i$ e $F_j^i$ polinomio omogeneo 
            di grado $j$. Dunque $V(I) = V(F^1,\ldots,F^r) = V(F_j^i : j \in \{0,\ldots,d_i\}, i \in \{1,\ldots,r\})$.
            \begin{definizione}
                Un ideale $I \ideal k[X_1,\ldots,X_{n+1}]$ si dice \emph{omogeneo} se per ogni $F \in I, F = \sum_{i=0}^d F_i$, dove $d$ è il grado di $F$ e 
                $F_i$ è un polinomio omogeneo di grado $i$ per ogni $i$, allora $F_i \in I$ per ogni $i$.
            \end{definizione}
            \begin{definizione}
                Sia $X \subseteq \P^n$ pongo $I(X) = \{F \in k[X_1,\ldots,X_{n+1}] : F(P)=0 \forall P \in X\}$ l'ideale associato ad $X$.
            \end{definizione} 
            \begin{osservazione}
                $I(X)$ è un ideale omogeneo per ogni $X \subseteq \P^n$.
            \end{osservazione}
            \begin{proposizione}
                Un ideale $I \ideal k[X_1,\ldots,X_{n+1}]$ è omogeneo se e solo se è generato da un numero finito di polinomi omogenei
            \end{proposizione}
            \begin{proof}
                Una dimostrazione di questo fatto è in \cite{fulton} Capitolo $4$, Paragrafo $2$.
            \begin{comment}
                Sia $I \ideal k[X_1,\ldots,X_{n+1}]$ omogeneo, e siano $F^1,\ldots,F^r$ polinomi che generano $I$, allora, se per ogni $1 \leq i \leq 
                r, d_i$ è il grado di $F^i$ e $F^i = \sum_{i=0}^{d_i} F_j^i$ con $F_j^i$ polinomio omogeneo di grado $j$, allora, $I = (F_j^i : j \in 
                \{0,\ldots,d_i\}, i \in \{1,\ldots,r\})$. \\
                Viceversa, siano $(F^\alpha)_{\alpha \in A}$, con $A$ insieme finito, dei generatori omogenei per $I$. Sia $F \in I$, e sia $F = 
                \sum_{i=m}^d F_i, F_i$ omogeneo di grado $i$. Siccome $F \in I$, esistono $B^{\alpha} \in k[X_1,\ldots,X_{n+1}]$ tali che $F = 
                \sum_{\alpha \in A} B^{\alpha}F^{\alpha}$. \\
                Denoto ora con $B_i^{\alpha}$ il monomio di grado $i$ del polinomio $B^{\alpha}$, e con $d_{\alpha}$ il grado del polinomio $F^{\alpha}$.\\
                Uguagliando i gradi segue che $F_m = \sum_{\alpha \in A} B_{m-d_{\alpha}}^{\alpha}F^{\alpha}$, da cui $F_m \in I$.\\
                Ne segue inoltre che $G = F - F_m \in I$ ed applicando lo stesso argomento a $G$, segue che $F_{m+1} \in I$. Per induzione si conclude.  
            \end{comment}
        \end{proof}
        \begin{proposizione}\label{prop:alg=clo}
            Unione finita di insiemi algebrici è un insieme algebrico. Intersezione arbitraria di insiemi algebrici è un insieme algebrico. $\emptyset,\P^n$ 
            sono insiemi algebrici.
        \end{proposizione}
        \begin{proof}
            Siano $S_1,S_2 \subseteq k[X_1,\ldots,X_{n+1}]$, dimostro che $V(S_1) \cup V(S_2) = V(S_1S_2)$: \\
            Sia $P \in V(S_1) \cup V(S_2)$, allora $P \in V(S_1)$ oppure $P \in V(S_2)$, cioè $F(P) = 0 \forall F \in S_1$ oppure $G(P) = 0 \forall 
            G \in S_2$. Ne segue che $\forall F \in S_1 \forall G \in S_2 \, FG(P) = F(P)G(P) = 0$, dunque $V(S_1) \cup V(S_2) \subseteq V(S_1S_2)$. \\
            Viceversa sia $P \in V(S_1S_2)$ e suppongo per assurdo che $P \notin V(S_1) \cup V(S_2)$, ovvero che esistano $F \in S_1, G \in S_2$ tali che 
            $F(P) \neq 0 \neq G(P)$, allora $0 = FG(P) = F(P)G(P)$, entrambi non nulli. Assurdo. \\
            Per induzione segue il risultato per famiglie finite. \\
            Sia ora $(S_{\alpha})_{\alpha \in A}$, con $A$ insieme arbitrario, tali che $S_{\alpha} \subseteq k[X_1,\ldots,X_{n+1}] \forall \alpha \in A$. 
            Allora, $\cap_{\alpha \in A} V(S_{\alpha})$ è un insieme algebrico: chiaramente, \begin{equation*}
                \bigcap_{\alpha \in A} V(S_{\alpha}) = V(\bigcup_{\alpha \in A}S_{\alpha})
            \end{equation*}
            e quest'ultimo è algebrico. Per dimostrare tale uguaglianza: \begin{equation*}
                P \in \bigcap_{\alpha \in A} V(S_{\alpha}) \Longrightarrow \forall \alpha \in A \forall F \in S_{\alpha} F (P) = 0 \Longrightarrow \forall 
                F \in \bigcup_{\alpha \in A}S_{\alpha} F(P) = 0
            \end{equation*}. Viceversa: \begin{equation*}
                P \in V(\bigcup_{\alpha \in A} S_{\alpha}) \Longrightarrow \forall F \in \bigcup_{\alpha \in A} S_{\alpha} F(P) = 0 \Longrightarrow \forall \alpha \in A \forall 
                F \in S_{\alpha} F(P) = 0
            \end{equation*}. \\
            $\emptyset = \{P \in \P^n : 1 = 0\} = V(1)$ e $\P^n = \{P \in \P^n : 0 = 0\} = V(0)$.
        \end{proof}
        \begin{osservazione}\label{obs:Zar}
            Gli insiemi algebrici, sia affini che proiettivi, sono dei chiusi per una topologia.
        \end{osservazione}
        Osservo ora che se $F \in k[X_1,\ldots,X_{n+1}]$ è un polinomio omogeneo, è ben definito, per ogni $i \leq n+1$ un polinomio in $n$ indeterminate, 
        detto affinizzato di $F$ rispetto alla $i$-esima coordinata omogenea: $F_i(X_1,\ldots,\hat{X_i},\ldots,X_{n+1}) = F(X_1,\ldots,X_{i-1},1,X_{i+1},\ldots,X_n)$. 
        $V = V(F), V_i = V(F_i) \forall i \leq n+1$, sono insiemi algebrici proiettivo e affini tali che $\forall i \leq n+1 \, \varphi_i(V \cap U_i) = V_i$.
        Un insieme algebrico proiettivo $V = V(S) \subseteq \P^n, S \subseteq k[X_1,\ldots,X_{n+1}]$ si dice \emph{irriducibile} se non è unione di insiemi algebrici più piccoli. Un 
        insieme algebrico irriducibile è detto \emph{varietà}. Vale anche in questo caso che $V$ è irriducibile se e solo se $I(V)$ è primo.\\
        Sia dunque $V$ un insieme algebrico irriducibile proiettivo e sia $I(V)$ il suo ideale primo associato. Allora considero l'anello $\gG_h(V) = 
        \frac{k[X_1,\ldots,X_{n+1}]}{I(V)}$. Siccome $I(V)$ è primo, $\gG_h(V)$ è un dominio ed è detto \emph{anello omogeneo associato} a $V$.\\
        Un elemento di $\gG_h(V)$ è detto omogeneo se è immagine, tramite la proiezione, di un polinomio omogeneo in $k[X_1,\ldots,X_{n+1}]$. Indico la proiezione con $\pi_V$.
        \begin{comment}
        \begin{proposizione}
            Ogni elemento $f \in \gG_h(V)$ si può scrivere in modo univoco come somma di omogenei tali che: $f = \sum_{i=0}^m f_i, f_i$ omogeneo di grado $i$.
        \end{proposizione}
        \begin{proof}
            Sia $F \in k[X_1,\ldots,X_{n+1}]$, tale che $\pi(F) = f$, allora, $F = \sum_{i=0}^m F_i, F_i$ polinomio omogeneo di grado $i$. Dunque $f = \pi(F) = \pi(\sum_{i=0}^m
            F_i) = \sum_{i=0}^m \pi(F_i) = \sum_{i=0}^m f_i$.\\
            Per quanto riguarda l'unicità siano $f = \sum_{i=0}^m g_i, g_i$ omogeneo di grado $i$. Allora, esistono $G_i \in k[X_1,\ldots,X_{n+1}],G_i$ polinomio omogeneo di grado 
            $i$ per ogni $i$, tali che $\pi(G_i) = g_i$ per ogni $i$, allora $F - \sum_{i=0}^m G_i = \sum_{i=0}^m F_i - G_i \in I(V)$ perché $\pi(F-\sum_{i=0}^m G_i) = 0$. Dunque, 
            $f_i = g_i$ per ogni $i$.  
        \end{proof}
        \end{comment}
        Siccome $\gG_h(V)$ è dominio, allora è ben definito il suo campo dei quozienti. Sia $k_h(V)$. Osservo ora che se $f,g \in \gG_h(V)$ sono omogenei dello stesso grado, 
        il rapporto $\frac{f}{g}$ induce una funzione sui punti di $V$ sui quali $g$ non si annulla, infatti, fissato un punto $P \in V$ tale che $g(P) \neq 0$, fissate delle coordinate 
        omogenee $\bar{x}$ per $P$, e detto $d$ il comune grado di $f$ e $g$, per ogni $\lambda \in k \setminus \{0\}$, quindi per ogni altra scelta di coordinate omogenee per $P$:
        \begin{equation*}
            \frac{f(\lambda x)}{g(\lambda x)} = \frac{\lambda^d f(x)}{\lambda^d g(x)} = \frac{f(x)}{g(x)}
        \end{equation*}
        Queste osservazioni portano a dare la seguente: \begin{definizione}
            Il campo delle funzioni su $V$ è $k(V) = \{z \in k_h(V) : z = \frac{f}{g}, f,g \text{ omogenei di stesso grado}\}$. Gli elementi di $k(V)$ sono detti \emph{funzioni 
            razionali su} $V$.
        \end{definizione}
        $k(V)$ è un sottocampo di $k_h(V)$. \\
        Siano ora $P \in V, z \in k(V)$ fissati; si dice che $z$ è definita in $P$, se esiste una coppia di omogenei dello stesso grado $f,g$, tali che $z = \frac{f}{g}$ e $g(P) 
        \neq 0$. Si definisce a questo punto $\cO_P(V) = \{z \in k(V) : z \text{ è definita in } P\}$. $\cO_P(V)$ è un anello locale, con ideale massimale $M_P(V) = \{
        z \in \cO_P(V) : z(P) = 0\}$.\\
        \begin{comment}
        Considero ora brevemente il caso di un multispazio, ovvero uno spazio del tipo $\P^{n_1} \times \cdots \times \P^{n_r} = X$, per opportuni $n_1,\ldots,n_r \in \N$.
        \begin{definizione}
            Un polinomio $F \in k[X_{1,1},\ldots,X_{n_1,1},\ldots,X_{1,r},\ldots,X_{n_r,r}] = Y$ si dice omogeneo se è omogeneo rispetto ad ogni famiglia di indeterminate.\\
            Un insieme algebrico in $X$ è $V(S)$, per un opportuno $S \subseteq Y$.
        \end{definizione}
        Valgono risultati e definizioni analoghi a quelli visti nel caso di insiemi affini e proiettivi.
    \end{comment}

    \newpage
    \section{Curve Algebriche Piane} \label{par:plane-curves}
        \subsection{Caso Affine}
            Siano $F,G \in k[X,Y]$, tali polinomi si dicono equivalenti se esiste $\lambda \in k, \lambda \neq 0$ tale che $F = \lambda G$. Questa relazione è un'equivalenza su 
            $k[X,Y]$.\\
            Definisco una \emph{curva piana affine} come una classe di equivalenza di polinomi non costanti rispetto alla relazione introdotta. Dunque posso definire il grado di una curva come 
            il grado di un polinomio (e quindi di tutti i polinomi) della classe di equivalenza.\\
            Sia quindi una curva fissata ed $F$ un rappresentante. Se $F = \prod F_i^{e_i}$, con gli $F_i$ non costanti, irriducibili ed a due a due non associati, allora, si dice che 
            $F_i$ è una \emph{componente della curva} $F$ \emph{di molteplicità} $e_i$. Se invece, $F$ è irriducibile, allora $V(F)$ è una varietà affine, dunque sono ben definiti $\gG(V(F)),
            k(V(F)),\cO_P(V(F))$, e si indicano con $\gG(F),k(F),\cO_P(F)$.\\
            Sia ora $F$ una curva e $P$ un suo punto. Si dice che $P$ è un \emph{punto semplice per} $F$ se $F_X(P) \neq 0$ o $F_Y(P) \neq 0$, dove $F_X,F_Y$ sono le derivate parziali di $F$. 
            In tal caso, la retta $F_X(P)(X-x_P) + F_Y(P)(Y-y_P) = 0$, è detta retta tangente ad $F$ in $P$.\\
            Suppongo ora che, a meno di una traslazione, $P = (0,0)$; allora $F = F_m + \cdots + F_n$, dove $n = \deg(F), F_i$ è polinomio omogeneo di grado $i$ in $k[X,Y]$, per ogni $i$ ed 
            $F_m \neq 0$. Si definisce la \emph{molteplicità della curva} $F$ \emph{nel punto} $P$ come $m$ e si scrive $m_P(F) = m$. Infine siccome, $F_m$ è omogeneo in due variabili, può essere 
            scritto nella forma $F_m = \prod_{i=1}^s L_i^{r_i}$, dove gli $L_i$ sono fattori lineari a due a due non associati. Gli $L_i$ sono le rette tangenti a $F$ in $P$ e ciascuna ha molteplicità 
            $r_i$.
            \begin{osservazione}
                $P \in F \iff m_P(F) > 0$. Se $P$ è semplice $m_P(F) = 1$. Se $m_P(F) > 1, P$ è detto punto \emph{multiplo}. 
            \end{osservazione}
            \noindent
            Il linguaggio degli anelli coordinati e degli anelli locali offre una diversa, ma equivalente caratterizzazione dei punti semplici e della molteplicità di una curva in un suo punto. 
            Userò la seguente notazione: per $G \in k[X,Y], g$ è la sua immagine in $\gG(F) = \frac{k[X,Y]}{(F)}$.
            \begin{proposizione}
                Un punto $P \in F$ è semplice se e solo se $\cO_P(F)$ è un DVR. Inoltre se $L$ è una retta per $P$ che non è tangente in $P$ a $F$, allora $\l \in \cO_P(F)$ è un parametro uniformizzante. 
            \end{proposizione}
            \begin{proof}
                Per la dimostrazione si veda \cite{fulton} Capitolo $3$, Paragrafo $2$.
            \end{proof}
            \begin{proposizione}
                Sia $P \in F, F$ irriducibile. Allora $m_P(F) = \dim_k \frac{M_P(F)^n}{M_P(F)^{n+1}}$ per $n$ sufficientemente grande.
            \end{proposizione}
            \begin{proof}
                Si veda \cite{fulton} Capitolo $3$, Paragrafo $2$.
            \end{proof}
            In particolare, da questo segue che la molteplicità di un punto dipende solo dal suo anello locale. Inoltre se $P$ è semplice, allora $\cO_P(F)$ è un DVR; sia $\ord_P^F$ la funzione ordine indotta su $k(F)$. \\
            Siano ora $F,G$ curve piane e $P \in \A^2$. Si definisce la \emph{molteplicità di intersezione} di $F$ e $G$ in $P$ come $I(P,F \cap G) = \dim_k \frac{\cO_P(\A^2)}{(F,G)}$. La moltepilicità di intersezione gode 
            delle seguenti proprietà: \begin{itemize}
                \item $I(P,F\cap G)$ esiste per ogni coppia di curve e per ogni punto;
                \item $I(P, F \cap G) \in \N$ se $F,G$ non hanno componenti comuni passanti per $P$, altrimenti, se $F,G$ hanno componenti comuni passanti per $P, I(P,F \cap G) = \infty$;
                \item $I(P,F \cap G) = 0 \iff P \notin F \cap G$, e $I(P,F \cap G)$ dipende solo dalle componenti di $F$ e $G$ passanti per $P$;
                \item Se $T$ è un cambio di coordinate affini, e $T(Q) = P$, allora $I(Q,F \cap G) = I(P, F^T \cap G^T)$;
                \item $I(P,F \cap G) = I(P,G \cap F)$;
                \item $I(P,F \cap G) \geq m_P(F)m_P(G)$ e vale l'uguaglianza se e solo se $F,G$ non hanno tangenti in $P$ in comune;
                \item Se $F = \prod_{i=1}^p F_i^{r_i}, G = \prod_{j=1}^q G_j^{s_j}$, allora $I(P,F \cap G) = \sum_{i,j}r_is_jI(P,F_i \cap G_j)$;
                \item $I(P,F \cap G) = I(P, F \cap (G + AF)), \, \forall A \in k[X,Y]$;
                \item Se $P$ è un punto semplice di $F$, allora, $I(P,F \cap G) = \ord_P^F(G)$;
                \item Se $F,G$ non hanno componenti comuni $\sum_{P \in \A^2} I(P,F \cap G) = \dim_k \frac{k[X,Y]}{(F,G)}$.
            \end{itemize}
        \subsection{Caso Proiettivo}
            Siano $F,G \in k[X,Y,Z]$ due polinomi omogenei non-costanti. Allora, $F,G$ si dicono equivalenti se esiste $\lambda \in k, \lambda \neq 0$, tale che $F = \lambda G$. Questa è un'equivalenza 
            tra i polinomi omogenei. Si definisce una \emph{curva piana proiettiva} come una classe di equivalenza. Il grado di una tale curva è il grado di un polinomio che la definisce.\\
            Osservo ora che se $F$ è una curva proiettiva e $P = [x,y,1]$ è un suo punto, allora, $(x,y) \in \A^2$ è un punto della curva affine $F_*$, definita come $F_*(X,Y) = F(X,Y,1)$, ovvero
            $F_*$ è \emph{l'affinizzato} di $F$. In particolare $\cO_P(F)$ è isomorfo a $\cO_{(x,y)}(F_*)$, dunque se $P \in U_3$ (o simmetricamente in $U_1$ o $U_2$), 
            risulta ben definita la molteplicità in $P$ di $F$, grazie alla teoria delle curve affini.\\
            In generale, dati dei punti $P_1,\ldots,P_n \in \P^2$, esiste una retta $L$ che non contiene alcuno di questi punti. Allora, a meno di un cambio di coordinate, posso supporre che questa retta 
            sia la retta $Z$, quindi i $P_i$ hanno coordinate $[x_i,y_i,1]$. \\
            Siccome c'è questa corrispondenza fra curve proiettive ed affini, risulta definita anche la molteplicità di intersezione di due curve proiettive in un punto. Una retta $L$ è detta \emph{tangente} 
            ad una curva $F$ in un punto $P$ se $I(P,L \cap F) \geq m_P(F)$. Un punto multiplo è detto \emph{ordinario} se ammette $m_P(F)$ tangenti distinte.\\
            \noindent
            Enuncio ora due teoremi che saranno molto importanti nel seguito. \\
            \begin{teorema}[di Bezout]
                Siano $F,G$ curve piane proiettive prive di componenti comuni. Sia $n = \deg(F), m = \deg(G)$. Allora $\sum_{P \in \P^2}I(P,F \cap G) = mn$.
            \end{teorema}
            \begin{proof}
                Si veda \cite{fulton} Capitolo $5$, Paragrafo $3$.
            \end{proof}
            \begin{definizione}
                Siano $F,G$ due curve passanti per $P$ prive di componenti comuni per $P$ e sia $H$ un'altra curva. Allora, si dice che \emph{le condizioni di Noether sono soddisfate in }$P$ 
                \emph{rispettivamente a} $F,G,H$ se $H_* \in (F_*,G_*) \subseteq \cO_P(\A^2)$.
            \end{definizione}
            \begin{teorema}[Fondamentale di Noether]
                Siano $F,G,H$ curve piane proiettive. Suppongo che $F,G$ non abbiano componenti comuni. Allora esistono $A,B \in k[X,Y,Z]$ omogenei tali che $H = AF + BG$ se e solo se le condizioni di Noether 
                sono soddisfatte in $P$, per ogni $P \in F \cap G$. 
            \end{teorema}
            \begin{proof}
                Si veda \cite{fulton} Capitolo $5$, Paragrafo $5$.
            \end{proof}

    \newpage
    \section{Varietà, Morfismi e Mappe Razionali}
        A questo punto risulta utile definire una topologia su $\P^n$(e su $\A^n$): la \emph{topologia di Zariski}, definita per ogni $U \subseteq \P^n$, come $U$ è un aperto se e solo se 
        $\P^n \setminus U$ è un insieme algebrico. Per l'Osservazione \ref{obs:Zar}, quella definita è effettivamente una topologia. 
        Sia ora $V$ un insieme algebrico irriducibile, e considero su $V$ la topologia indotta dalla topologia di Zariski. Siano $U_1,U_2 \subseteq V$ due aperti, allora, $U_1 
        \cap U_2 \neq \emptyset$ perché altrimenti, $V = (V \setminus U_1) \cup (V \setminus U_2)$, sarebbe riducibile. Ne segue che per ogni coppia di punti distinti $P,Q \in 
        V$ i loro intorni non sono mai disgiunti. Ne segue che $\P^n$ con la topologia di Zariski non è uno spazio Hausdorff.\\
        %La topologia di Zariski è ben definita anche nei multispazi.
        \begin{definizione}
            Sia $V \subseteq \P^n$ un insieme algebrico irriducibile, e sia $X \subseteq V$ un aperto. $X$ è detto varietà. Analogamente risultano definite le varietà per insiemi affini. %ed in mutlispazi.
        \end{definizione}
        Analogamente a quanto visto per gli insiemi algebrici, possiamo definire le funzioni razionali su $X$ varietà, come $k(X) = \{f_{\restriction_X} : f \in k(V)\}$, ed analogamente, 
        per $P \in X, \cO_P(X) = \{f \in k(X) : f \text{ è definita in } P\}$.\\
        Se $U \subseteq X$ è aperto, allora $U$ è aperto in $V$, dunque è una varietà ed è detto \emph{sottovarietà aperta} di $X$.\\
        Sia ora $Y \subseteq X$ un chiuso, allora, $Y$ si dice irriducibile se non è unione di due suoi sottoinsiemi propri e chiusi in $X$. Se $Y$ è irriducibile, allora, detta $\bar{Y}$ 
        la sua chiusura in $V$, $Y = \bar{Y} \cap X$ è un aperto di $\bar{Y}$, quindi è una varietà di $\bar{Y}$, ed è detta \emph{sottovarietà chiusa} di $X$.
        Analoghe definizioni valgono nel caso affine. \\
        Sia ora $U \subseteq X$ un aperto non vuoto; definisco $\gG(U) = \{f \in k(X) : f \text{ è definita in ogni punto } P \in U\} = \cap_{P \in U} \cO_P(X)$. \\
        Considero dunque l'anello $\cI(U,k)$ delle mappe da $U$ a $k$.
        \begin{lemma}\label{lem:gamma-cont}
            Sia $X$ una varietà proiettiva e sia $U$ un suo sottoinsieme aperto. Sia $z \in \gG(U)$ tale che $z(P) = 0$ per ogni $P \in U$. Allora $z = 0$.
        \end{lemma}
        \begin{proof}
            Sia $z \in \gG(U)$, allora $z \in k(X)$ e $z$ è definita in ogni punto di $U$; cioè $z = \frac{f}{g}, f,g \in \gG_h(X)$ omogenei dello stesso grado, con $g(P) \neq 0 \forall P \in U$.
            Allora $f(P) = 0 \forall P \in U$. \\
            Dimostro ora che $z = \frac{f}{g} : U \to k$ è una funzione continua se considero su $U$ la topologia indotta dalla topologia di Zariski e su $k$ la topologia di Zariski, una volta 
            identificato $k = \A^1(k) = \A^1$. Sia un chiuso $A \subseteq \A^1$ un chiuso, allora, è un insieme finito. Dunque siccome le antiimmagini commutano con le unioni è sufficiente dimostrare 
            che $z^{-1}(a)$ è un chiuso per ogni $a \in \A^1 = k$. \begin{multline*}
                z^{-1}(a) = \{P \in U : z(P) = a\} = \{P \in U : f(P) -ag(P) = 0\} = \\ = \{P \in U : F(P) - aG(P) = 0\} = V(F-aG) \cap U
            \end{multline*}
            dove $F,G$ sono polinomi omogenei che vengono mappati in $f,g$ nel quoziente $\gG_h(X)$. In particolare, $z^{-1}(a)$ è algebrico quindi chiuso.\\
            Infine essendo quindi $z$ continua, ed essendo $U$ denso per la topologia di Zariski, $z(X) = z(\bar{U}) \subseteq \bar{0} = 0$. Ne segue $z = 0$.
        \end{proof}
        Siccome quindi la mappa $\gG(U) \to \cI(U,k)$ è una mappa iniettiva, posso identificare $\gG(U)$ con la sua immagine. \\
        D'ora in poi con varietà intenderò sia insiemi algebrici proiettivi (o affini) irriducibili, sia quelle che ho chiamato sottovarietà (aperte e chiuse) sia affini che proiettive. 
        Siano quindi $X,Y$ varietà e sia $\varphi: X \to Y$ una mappa insiemistica. Allora è ben definito l'omomorfismo d'anelli, $\tilde{\varphi}: \cI(Y,k) \to \cI(X,k)$ definito per 
        ogni funzione $f \in \cI(Y,k)$ da $\tilde{\varphi}(f) = f \circ \varphi$.
        \begin{definizione}
            Una mappa $\varphi: X \to Y$, con $X,Y$ varietà è detta \emph{morfismo}, se:\begin{enumerate}
                \item $\varphi$ è continua rispetto alle topologie di Zariski su $X$ e $Y$;
                \item per ogni aperto $U \subseteq Y$ e per ogni $f \in \gG(U)$, allora $f \circ \varphi \in \gG(\varphi^{-1}(U))$.
            \end{enumerate}
            Un \emph{isomorfismo} è un morfismo $\varphi$ che è invertibile e $\varphi^{-1}$ è un morfismo.
        \end{definizione}
        \begin{definizione}
            Siano $V \subseteq \A^n, W \subseteq \A^m$; una mappa $p : V \to W$ è una \emph{mappa polinomiale} se $p = (p_1,\ldots,p_m)$ e $p_i \in k[X_1,\ldots,X_n] \, \forall i$. 
        \end{definizione}
        D'ora in poi mi userò la seguente nomenclatura: dirò che una varietà è affine se è isomorfa ad una varietà in uno spazio affine.
        \begin{proposizione}\label{prop:morph}
            Siano $X$,$Y$ varietà affini. Allora esiste una corrispondenza iniettiva fra morfismi $\varphi : X \to Y$ ed omomorfismi $\tilde{\varphi} : \gG(Y) \to \gG(X)$. In particolare, 
            un morfismo di $X \subseteq \A^n$ in $Y \subseteq \A^m$ è equivalente ad una mappa polinomiale.
        \end{proposizione}
        \begin{proof}
            Si veda \cite{fulton} Capitolo $6$, Paragrafo $3$.
        \end{proof}
        \begin{comment}
        \begin{proof}
            Siano $X \subseteq \A^n,Y \subseteq \A^m$ varietà. Allora, la dimostrazione discende dai seguenti fatti: \begin{itemize}
                \item una mappa polinomiale è un morfismo;
                \item un morfismo $\varphi : X \to Y$ induce un omomorfismo $\tilde{\varphi} : \gG(Y) \to \gG(X)$;
                \item ogni omomorfismo $\tilde{\varphi} : \gG(Y) \to \gG(X)$ è indotto da una mappa polinomiale;
            \end{itemize}
            Dimostro che una mappa polinomiale tra $X$ e $Y$ è un morfismo: siano $p_1,\ldots,p_m \in k[X_1,\ldots,X_n]$ tali che la mappa polinomiale associata $p : X \to Y$ sia ben definita. $p$ è un morfismo. \\
            $p$ è continua per la topologia di Zariski: sia $A \subseteq Y$ un chiuso, allora esiste $S \subseteq k[Y_1,\ldots,Y_m]$ tale che $A = V(S) \cap Y$; \begin{multline*}
                p^{-1}(A) = \{P \in X : p(P) \in A\} =  \\
                = \{P \in X : F(p(P)) = 0  \forall F \in S\} = V(S_p) \cap  X
            \end{multline*}
            Dove $S_p$ è l'insieme definito come  $\{F(p_1(X_1,\ldots,X_n),\ldots,p_m(X_1,\ldots,X_n)) : F \in S\}$; siccome $p_1,\ldots,p_m$ sono polinomi e gli elementi di $S$ anche , segue che $S_p \subseteq 
            k[X_1,\ldots,X_n]$, da cui $p^{-1}(A)$ è un chiuso. \\
            Sia $f \in \gG(Y)$ allora $f$ è immagine in un anello quoziente di un polinomio $F$, allora $\tilde{p}(f) = f \circ p$ è l'immagine nello stesso quoziente del polinomio $F(p_1(X_1,\ldots,X_n),\ldots,p_m(X_1,\ldots,X_n))$ 
            dunque è in $\gG(X)$, perché la mappa polinomiale $p$ è ben definita. Questo dimostra che $p$ è un morfismo.\\
            Sia ora $\varphi : X \to Y$ un morfismo; allora è ben definita la mappa insiemistica $\tilde{\varphi} : \gG(Y) \to \gG(X)$. Dimostro che $\tilde{\varphi}$ è omomorfismo. \\
            $\tilde{\varphi}(1) = 1 \circ \varphi = 1 \in \gG(X)$. $f,g \in \gG(Y)$, allora , per ogni $P \in X$: \begin{equation*}
                \tilde{\varphi}(f+g)(P) = (f+g) \circ \varphi (P) = f(\varphi(P)) + g(\varphi(P)) = \tilde{\varphi}(f)(P) + \tilde{\varphi}(g)(P) 
            \end{equation*}
            Analogamente: \begin{equation*}
                \tilde{\varphi}(fg)(P) = (fg) \circ \varphi (P) = f(\varphi(P)g(\varphi(P))) = \tilde{\varphi}(f)(P)\tilde{\varphi}(g)(P)
            \end{equation*}
            Questo prova che $\tilde{\varphi}$ è un omomorfismo. \\ 
            Dimostro che ogni per ogni omomorfismo $\tilde{\varphi} : \gG(Y) \to \gG(X)$ esiste una mappa polinomiale $\varphi : X \to Y$ tale che $\tilde{\varphi}$ sia l'omomorfismo associato a $\varphi$.
            Sia $\tilde{\varphi}$ fissato e sia, per ogni $i \in {1,\ldots,m}$, $p_i \in k[X_1,\ldots,X_n]$ tale che $\tilde{\varphi}(\pi_Y(Y_i)) = \pi_X(p_i)$. Allora $\varphi : X \to Y$ che associa a $P, \varphi(P) = 
            (p_1(P),\ldots,p_m(P)) \in Y$ è una mappa polinomiale di $X$ in $Y$ e, per costruzione è tale che $\tilde{\varphi}(f) = f \circ \varphi$ per ogni $f \in \gG(Y)$. \\
            Sia ora $\varphi : X \to Y$ un morfismo e sia $\tilde{\varphi} : \gG(Y) \to \gG(X)$ l'omomorfismo indotto. Per quanto appena visto esiste una mappa polinomiale $p : X \to Y$ tale che l'omomorfismo 
            indotto da $p$ è $\tilde{\varphi}$. Allora, per ogni $f \in \gG(Y), f \circ \varphi = f \circ p$; quindi: \begin{equation*}
                \forall f \in \gG(Y) \forall F \in \pi_Y^{-1}(f) \pi_Y(F) \circ \varphi = \pi_Y(F) \circ p = \pi_X(F \circ p)
            \end{equation*}
            Quindi le funzioni $F \circ \varphi, F \circ p$, come funzioni da $X$ in $k$, sono uguali per ogni $F \in k[Y_1,\ldots,Y_m]$, ne segue $\varphi = p$.
        \end{proof}
    \end{comment}
    \begin{esempio}
        Fissato $V \subseteq \P^n$, una varietà, $U_i,\varphi_i, i \{1,\ldots,n+1\}$ gli aperti e le mappe definiti in \ref{par:aff-proj}. Siano inoltre $V_i = V \cap U_i, 
        \tilde{V_i} = \varphi_i(V_i)$, allora $\varphi_i : V_i \to \tilde{V_i}$ è isomorfismo per ogni $i$, quindi ogni varietà proiettiva è unione di sottovarietà aperte isomorfe a varietà affini.
    \end{esempio}
    \begin{definizione}
        Sia $K$ un'estensione di $k$ generata aggiungendo a $k$ un numero finito di elementi. Si dice grado di trascendenza di $K$ su $k$, e si denota con $\text{tr.deg}_kK$, il più piccolo intero $n$, tale che esistono 
        $x_1,\ldots,x_n \in K$, tali che $K$ è algebrico su $k(x_1,\ldots,x_n)$. In tal caso si dice che $K$ è un campo di funzioni algebriche in $n$ variabili su $k$.
    \end{definizione}
    \begin{proposizione} \label{prop:fun-fields}
        Sia $K$ un campo di funzioni algebriche in una variabile su $k$, tale che per ogni $t \in K$, tale che $K$ è algebrico su $k(t)$, allora l'estensione $\frac{K}{k(t)}$ è finita e sia $x \in K \setminus k$. Allora: \begin{enumerate}
            \item $K$ è algebrico su $k(x)$;
            \item Esiste un elemento $y \in K$ tale che $K = k(x,y)$.
        \end{enumerate}
    \end{proposizione}
    \begin{proof}
        Sia $t \in K$ tale che $K$ è estensione algebrica di $k(t)$; allora esiste un polinomio $F \in k(t)[X]$ tale che $F(t,x) = 0$. In particolare, siccome $x$ non è algebrico su $k$, perché $k$ è algebricamente chiuso, allora $t$ compare 
        in $F(t,x)$. Allora, moltiplicando gli eventuali denominatori, posso concludere che esiste $G \in k(x)[T]$ tale che $G(x,t) = 0$, da cui $t$ è algebrico su $k(x)$, ma allora $k(x,t)$ è algebrico su $k(x)$ e di conseguenza lo è $K$. \\
        Siccome $K$ è algebrico su $k(x)$, allora l'estensione è algebrica e finita, quindi ammette elemento primitivo, ovvero esiste $y \in K$ tale che $K = k(x,y)$.
    \end{proof}
    Se $X$ è una varietà, allora, $k(X)$ è un'estensione di $k$ finitamente generata. Si definisce allora $\text{dim}(X) = \text{tr.deg}_kk(X)$. Una varietà di dimensione $1$ è detta curva.
    \begin{osservazione}
        Una curva secondo questa definizione è irriducibile, mentre una curva piana definita come in \ref{par:plane-curves} può essere riducibile.
    \end{osservazione}
    \begin{proposizione}
        \begin{enumerate}
            \item Se $U$ è una sottovarietà aperta di $X$, allora $\dim(U) = \dim(X)$;
            \item Se $V$ è la chiusura proiettiva di una varietà affine $V'$, allora $\dim(V) = \dim(V')$;
            \item Una varietà ha dimensione zero se e solo se è un punto;
        \end{enumerate}
    \end{proposizione}
    \begin{proof}
        I primi due punti discendono dal fatto che i campi di funzioni coincidono.\\
        Sia ora $V$ una varietà di dimensione zero: per i primi due punti possiamo supporre sia affine; allora siccome $k(V)$ è algebrico su $k$, ma $k$ è algebricamente chiuso, segue che $k(V) = k$. 
        In particolare $\gG(V) = k$, quindi i resti modulo $I(V)$ sono solo costanti, quindi $I(V)$ è generato da $n$ polinomi di primo grado linearmente indipendenti su $k$, che si annullano in $V$, 
        ma quindi $V$ è un unico punto in $\A^n$. Il viceversa è ovvio.
    \end{proof} 
    \begin{definizione}
        Siano $X,Y$ varietà, due morfismi $f_1 : U_1 \to Y, f_2 : U_2 \to Y$, con $U_1,U_2 \subseteq X$ aperti, si dicono equivalenti se le loro restrizioni a $U_1 \cap U_2$ coincidono.
    \end{definizione}
    Siccome $U_1 \cap U_2$ è denso in $X$, $f_1,f_2$ sono determinati dalle loro restrizioni su $U_1 \cap U_2$. Questa relazione è effettivamente una relazione di equivalenza fra i morfismi. Una classe di equivalenza di morfismi è una 
    coppia $(U,f)$ dove $U \subseteq X, U = \cup_{\alpha} U_{\alpha}, U_{\alpha}$ dominio di un singolo morfismo, $f : U \to Y$ definita da $P \in U \Longrightarrow P \in U_{\alpha} \exists \alpha, f(P) = f_{\alpha}(P)$, con $f_{\alpha}$ 
    morfismo di dominio $U_{\alpha}$. Siccome morfismi equivalenti coincidono sulle intersezioni dei rispettivi domini, la definizione di $f$ è ben posta. $f$ è detta \emph{mappa razionale} ed $U$ è il suo dominio.
    \begin{definizione}
        Una mappa razionale $f : U \to Y, U \subseteq X$ è detta \emph{dominante} se $f(U)$ è denso in $Y$.\\
        Siano $A,B$ anelli locali tali che $A \leq B$; si dice che $B$ \emph{domina} $A$, se l'ideale massimale di $B$ contiene l'ideale massimale di $A$. 
    \end{definizione}
    \begin{proposizione}\label{prop:techn1}
        Siano $X,Y$ varietà e sia $F : X \to Y$ una mappa razionale dominante. Siano $U \subseteq X, V \subseteq Y$ aperti tali che $f : U \to V$ è un morfismo che rappresenta $F$. Allora: \begin{enumerate}
            \item l'omomorfismo indotto $\tilde{f} : \gG(V) \to \gG(U)$ è iniettivo, quindi si estende unicamente ad un omomorfismo di $k(V) = k(Y)$ in $k(U) = k(X)$; inoltre è indipendente dalla scelta di $f$, quindi si denota con $\tilde{F}$;
            \item se $P$ è nel dominio di $F, F(P) = Q$, allora $\cO_P(X)$ domina $\tilde{F}(O_Q(Y))$; viceversa se $\cO_P(X)$ domina $\tilde{F}(\cO_Q(Y))$ per oppurtuni $P \in X, Q \in Y$, allora $P$ è nel dominio di $F$ e $F(P) = Q$;
            \item ogni omomorfismo di $k(Y)$ in $k(X)$ è indotto da un un'unica mappa razionale dominante di $X$ in $Y$.
         \end{enumerate}
    \end{proposizione}
    \begin{proof}
        Si veda \cite{fulton} Capitolo $6$, Paragrafo $6$.
    \end{proof}
    \begin{comment}
    \begin{proof}
        Siano $f_1,f_2 \in \gG(Y)$ tali che $\tilde{f}(f_1) = \tilde{f}(f_2)$, allora vale che per ogni $P \in f(U), f_1(P) = f_2(P)$, ma $f(U)$ è denso per ipotesi e, per la dimostrazione del \ref{lem:gamma-cont} le funzioni $f_1 = f_2$. 
        Siccome l'omomorfismo di anelli è iniettivo, può essere esteso ad un omomorfismo di campi. \\ Siano $f : U_f \to V_f,g : U_g \to V_g$ due morfismi che rappresentano $F$, allora considero $h : U_h = U_f \cap U_g \to V_h = f(U_f \cap U_g) = g(U_f \cap U_g)$.
        $h$ è ben definita perché $f,g$ sono equivalenti, inoltre, $\tilde{f} = \tilde{h}$ perché $h$ è restrizione di $f$ e $\gG(U_f) = \gG(U_f)$. Analogamente per $\tilde{g} = \tilde{h}$. \\
        Sia $P$ nel dominio di $F$ e sia $F(P) = Q$. $\cO_P(X)$ è anello locale e siccome $\tilde{F}$ è omomorfismo, anche $\tilde{F}(\cO_Q(Y))$ lo è. Se $\tilde{\varphi} \in \tilde{F}(\cO_Q(Y))$, allora $\tilde{\varphi} = \varphi \circ f$, dove $f$ 
        rappresenta $F$ e $\varphi \in \cO_Q(Y)$, allora, $\tilde{\varphi}(P) = \varphi(f(P)) = \varphi(Q)$, quindi $\tilde{\varphi}$ è definita in $P$, da cui $\tilde{F}(\cO_Q(Y)) \leq \cO_P(X)$. \\
        Sia $M_P$ l'ideale massimale di $\cO_P(X)$ e $\tilde{M_Q}$ l'ideale massimale di $\tilde{F}(\cO_Q(Y))$. Allora sia $\tilde{\gvf} \in \tilde{F}(\cO_Q(Y))$, allora esiste $\gvf \in \cO_Q(Y)$ non invertibile tale che $\tilde{\gvf} = \gvf \circ f$, dove 
        $f$ rappresenta $F$. Allora, $\tilde{\gvf}(P) = \gvf(Q) = 0$, dunque è non invertibile. $\tilde{\gvf} \in M_P$. \\
        Viceversa siano $P \in X, Q \in Y$, tale che $\cO_P(X)$ domina $\tilde{F}(\cO_Q(Y))$. Considero intorni affini $V$ di $P$, $W$ di $Q$. Sia $\gG(W) = k[y_1,\ldots,y_n]$, allora $\tilde{F}(y_i) = \frac{a_i}{b_i}, a_i,b_i \in \gG(V), b_i(P) \neq 0 \forall i$. 
        Sia $b = b_1 \cdots b_n$, allora $\tilde{F}(\gG(W)) \subseteq \gG(V_b) = \gG(V)[\frac{1}{b}], V_b = \{P \in V : b(P) \neq 0\}$. \\
        Ne segue che l'omomorfismo $\tilde{F} : \gG(W) \to \gG(V_b)$ è ben definito, dunque per la Proposizione \ref{prop:morph} è indotta da un unico morfismo $f : V_b \to W$. \\
        Sia quindi $g \in \gG(W)$ che si annulla in $Q$, allora $\tilde{F}(g)$ si annulla in $P$, e di conseguenza $f(P) = Q$. \\
        Suppongo ora $X,Y$ affini, allora per ogni omomorfismo $\gvf : k(Y) \to k(X), \gvf(\gG(Y)) \subseteq \gG(X_b)$, per un opportuno $b \in \gG(X)$, allora $\gvf$ è indotta da un morfismo $f : X_b \to Y$, e $f(X_b)$ è denso. \\
        Dimostro la densità di $f(X_b)$: $\gvf$ iniettivo perché estensione di $\tilde{F}$, che è dominante quindi per ogni coppia di funzioni $g,h \in k(Y), g \circ \gvf = h \circ \gvf \Longrightarrow g = h$. Ciò avviene se e solo se 
        l'uguaglianza di $g,h$ su $f(X_b)$ implica l'uguaglianza di $g,h$ per ogni coppia di funzioni, ma questo è vero solo se $X_b$ è denso. 
    \end{proof}
    \end{comment}
    Una mappa razionale di $X$ in $Y$ è detta birazionale se esistono degli aperti $U \subseteq X, V \subseteq Y$ ed un isomorfismo $f : U \to V$ che rappresenta $F$. Due varietà tra cui esiste una mappa birazionale, si dicono birazionalmente equivalenti. Ad esempio ogni 
    varietà è birazionalmente equivalente ad ogni sua sottovarietà aperta. 
    \begin{proposizione}
        Due varietà sono birazionalmente equivalenti se e solo se i loro campi di funzioni sono isomorfi
    \end{proposizione}
    \begin{proof}
        Che due varietà birazionalmente equivalenti abbiano campi di funzioni isomorfi è ovvio. \\
        Viceversa, se $\gvf : k(Y) \to k(X)$ è un isomorfismo, allora, per la dimostrazione della Proposizione \ref{prop:techn1} $\gvf(\gG(X)) \subseteq \gG(Y_b)$ per un opportuno $b \in \gG(Y)$ e $\gvf^{-1}(\gG(Y)) \subseteq \gG(X_d)$ per un opportuno $d \in \gG(X)$. 
        Allora $\gvf$ si restringe ad un isomorfismo tra $\gG((Y_b)_{\gvf^{-1}(d)})$ e $\gG((X_d)_{\gvf(b)})$, che è generato da un unico morfismo $f : (X_d)_{\gvf(b)} \to (Y_b)_{\gvf^{-1}(d)}$.
    \end{proof}
    \begin{corollario}
        Ogni curva è birazionalmente equivalente ad una curva piana.
    \end{corollario}
    \begin{proof}
        Sia $V$ una curva allora, per la proposizione \ref{prop:fun-fields}, esistono $x,y \in k(V)$ tali che $k(V) = k(x,y)$. Considero perciò il naturale omomorfismo d'anelli da $k[X,Y]$ in $k[x,y]$, e sia $I$ il suo nucleo. In particolare, essendo $V$ irriducibile, $I$ è primo, 
        dunque $V' = V(I) \subseteq \A^2$ è una varietà. Inoltre, $\gG(V') = \frac{k[X,Y]}{I}$ è isomorfo a $k[x,y]$, quindi $k(V')$ è isomorfo, a $k(V)$, quindi per la proposizione precedente le due varietà sono birazionalmente equivalenti. Per vedere che $V'$ è una curva è 
        sufficiente osservare che essendo $k(V),k(V')$ isomorfi, $\text{dim}V' = \text{dim}V = 1$.
    \end{proof}
    \begin{esercizio} \label{ex:dm}
        Siano $C,C'$ curve e sia $F$ una mappa razionale tra $C$ e $C'$. Allora $F$ è dominante oppure è costante. Inoltre se $F$ è dominante, $k(C)$ è un'estensione algebrica finita di $\tilde{F}(k(C'))$.
    \end{esercizio}
    \begin{proof}
        Se $F$ è dominante allora non c'è niente da dimostrare. \\
        Sia quindi $F$ non dominante, e sia $f : U \to V$ morfismo che rappresenta $F$. Allora, siccome $F$ non è dominante, $f(U)$ è non vuoto e non è denso in $C'$. Esiste perciò $\emptyset \neq V \subseteq C'$ aperto tale che $f(U) \cap V = \emptyset \Longrightarrow f(U) \subseteq C' 
        \setminus V$ che è algebrico, quindi è una sottovarietà chiusa di $C'$ e siccome $f(U) \neq \emptyset$ è non banale ovvero è un punto. Ne segue $f(U) = \{P\}$ per un opportuno $P \in C'$. Essendo $U$ denso in $C$ segue la tesi. \\
        Sia ora $F$ dominante allora $\tilde{F}$ è un omomorfismo non banale di campi, dunque $L = \tilde{F}(k(C'))$ è isomorfo a $k(C')$. Sia $L$ che $k(C)$ sono campi di funzione in una variabile su $k$. Dunque per la proposizione \ref{prop:fun-fields} esistono $x,y \in k(C), t,s \in L$, 
        nessuno dei quattro in $k$ tali che $k(C) = k(x,y), L = k(t,s)$. Sempre per la Proposizione \ref{prop:fun-fields} $k(C)$ è estensione algebrica finita di $L$: basta aggiungere $x,y$, quando non già presenti.
    \end{proof}

\newpage
\section{Scoppiamento di punti affini e proiettivi, Trasformazioni quadratiche e Modello non-singolare}
    Sia $C$ una curva arbitraria e sia $P$ un suo punto. $P$ è detto punto semplice se  $\cO_P(C)$ è un DVR.\\
    Sia quindi $\ord_P^C$ la funzione d'ordine su $k(C)$ associata a $\cO_P(C)$. Se ogni punto di $C$ è semplice, la curva è detta \emph{non-singolare}.
    \begin{definizione}
        Siano $k \leq K$ campi; un sottoanello $A$ di $K$ è detto \emph{anello locale di} $K$, se $A$ è un anello locale, $K$ è il campo dei quozienti di $A$ e $k \leq A$. Analogamente si dice che $A$ è un  \emph{anello di valutazione discreta di} $K$ se $A$ è un anello locale 
        di $K$ ed è un DVR.
    \end{definizione}
    \begin{proposizione} \label{prop:techn2}
        Sia $C$ una curva proiettiva e sia $K = k(C)$. Sia inoltre $L$ un campo contenente $K$ ed $R$ un DVR di $L$ che non contiene $K$. Allora esiste un unico punto $P \in C$ tale che $R$ domina $\cO_P(C)$.
    \end{proposizione}
    \begin{proof}
        Si veda \cite{fulton} Capitolo $7$, Paragrafo $1$.
    \end{proof}
    \begin{comment}
    \begin{proof}
        Posso supporre che $C$ sia una sottovarietà chiusa di $\P^n$, e che, a meno di un cambio di coordinate, $C \cap U_i \neq \emptyset$ per ogni $i$, dove $U_i$ sono gli aperti di $\P^n$ definiti nel Paragrafo \ref{par:aff-proj}. \\
        Allora, esistono $x_1,\ldots,x_n \in \gG_h(C)$, tali che $\gG_h(C) = \frac{k[X_1,\ldots,X_{n+1}]{I(C)}} = k[x_1,\ldots,x_{n+1}], x_i \neq 0 \forall i$.\\
        Sia ora $N = \,ax_{i,j}\ord(\frac{x_i}{x_j})$ e , a meno di riordinare le coordinate, posso suppore che esista $j \leq n$ tale che $N = \ord(\frac{x_j}{x_{n+1}})$. Ne segue che per ogni $i$: \begin{equation*}
            \ord(\frac{x_i}{x_{n+1}}) = \ord(\frac{x_i}{x_j}\frac{x_j}{x_{n+1}}) = N - \ord(\frac{x_j}{x_i}) \geq 0
        \end{equation*}
        Sia  ora $C_{n+1}$ la curva affine associata a $C \cap U_{n+1}$, allora, $\gG(C_{n+1})$ si può identificare con $k[\frac{x_1}{x_{n+1}},\ldots,\frac{x_n}{x_{n+1}}]$, e per quanto visto sopra, $\gG(C_{n+1}) \leq R$.\\
        Sia ora $M$ l'ideale massimale di $R$ e sia $J = M \cap \gG(C_{n+1})$. \\
        $J$ è un ideale primo: è sottoinsieme di un ideale chiuso per somma, per prodotto per un elemento di $\gG(C_{n+1})$ e non è vuoto ad esempio, $0 \in J$. Dunque è un ideale. Inoltre, se $f,g \in \gG(C_{n+1}),fg \in J \ord(fg) = \ord(f) + \ord(g) > 0$, 
        quindi almeno uno dei due ordini deve essere positivo, sia quello di $f$. Dunque $f \in J$. \\
        Allora, esiste una sottovarietà chiusa di $V \subseteq C_{n+1}$ associata a $J$; siccome $\dim(C_{n+1}) = 1$, se $\dim(V) = 1$, allora $V = C_{n+1}$, ma allora $J = (0)$, e gli elementi non nulli di $\gG(C_{n+1})$ sono invertibili in $R$, dunque $K \subseteq R$, ma questo è assurdo.\\
        Allora, $\dim(V) = 0$, cioè $V$ è un punto $P$. \\
        Sia $f \in \cO_P(C_{n+1})$, allora, $f \in k(C_{n+1})$, i cui elementi sono quozienti di elementi di $\gG(C_{n+1})$, quindi in questo anello esistono $g,h$ definite in $P$ con $h(P) \neq 0$ e $f(P) = \frac{g(P)}{h(P)}$; allora, siccome $\gG(C_{n+1}) \leq R$ e $h(P) \neq 0, h \notin M$, perché se 
        così fosse, $h \in M \cap \gG(C_{n+1})$, dunque $h(P) = 0$, assurdo. Dunque $h(P) \notin M \Longrightarrow h$ è invertibile in $R$: \begin{equation*}
            \ord(f) = \ord(\frac{g}{h}) = \ord(g) - \ord(h) = \ord(g) \geq 0
        \end{equation*}
        Allora $M_P(C_{n+1}) = \{f \in \cO_P(C_{n+1}) : f(P) = 0\}$, ma con le notazioni precedenti questo accade se e solo se $g(P) = 0 \iff g \in M \iff \ord(f) = \ord(g) > 0$. Questo prova che $R$ domina $\cO_P(C_{n+1}) = \cO_P(C)$.\\
        $P$ è unico, infatti $R$ dominasse anche $\cO_Q(C_{n+1}), Q \neq P$, allora, esiste $f \in k(C_{n+1})$ tale che $f \in M_P(C_{n+1}), \frac{1}{f} \in \cO_Q(C_{n+1})$, dunque $\ord(f) > 0,\ord(\frac{1}{f}) \geq 0$, assurdo. \\
        Una tale funzione esiste sempre perché: $f \in k(C_{n+1})$ è quoziente di elementi di $\gG(C_{n+1})$, quindi siccome $P,Q$ sono distinti esiste $f \in \gG(C_{n+1})$, che si annulla in $P$ e non in $Q$, quindi $f \in k(C_{n+1})$ è una tale funzione.
    \end{proof}
    \end{comment}
    \begin{corollario}
        Sia $F$ una mappa razionale da una curva $C$ ad una curva proiettiva $C'$; allora i punti semplici di $C$ sono nel dominio di $F$.
    \end{corollario}
    \begin{proof}
        Se $F$ non è domininante allora è costante (Esercizio \ref{ex:dm}), quindi è definita su ogni punto di $C$. \\
        Sia quindi $F$ dominante e $P \in C$ un punto semplice. Allora, posto $R = \cO_P(C), L = \tilde{F}(k(C'))$, per la Proposizione \ref{prop:techn1}, se $R$ domina $\tilde{F}(\cO_Q(C'))$ per un $Q \in C'$, allora $P$ è nel dominio di $F$. Ma siccome $F$ è dominante 
        $\tilde{F}$ è isomorfismo tra $k(C')$ ed $L$, quindi $R$ domina $\tilde{F}(\cO_Q(C'))$ per un oppurtuno $Q$ se $L \not\subseteq R$. \\
        Se per assurdo fosse $L \subseteq R \subseteq k(C)$, essendo $\frac{k(C)}{L}$ algebrica finita per l'Esercizio \ref{ex:dm}, segue che $R$ è un campo per l'Esercizio \ref{ex:fields}. Ma questo è assurdo perché un DVR non è un campo.
    \end{proof}
    \begin{corollario}
        Sia $C$ una curva proiettiva non-singolare, $K = k(C)$. Allora i DVR di $K$ sono tutti e soli gli $\cO_P(C)$.
    \end{corollario}
    \begin{proof}
        Che gli $\cO_P(C)$ siano DVR di $K$ è ovvio. Viceversa se $R$ è un DVR di $K$, allora per la Proposizione \ref{prop:techn2} $R$ domina un unico $\cO_P(C)$. Quindi siamo nella situazione di due DVR con uno che domina l'altro inclusi nello stesso campo, che per entrambi è il campo dei quozienti. 
        Ne segue $R = \cO_P(C)$. L'inclusione non banale si dimostra così: $r \in R \subseteq K \Longrightarrow r = \frac{g_1}{g_2}, g_2 \neq 0, g_1,g_2 \in \cO_P(C)$; se $g_2$ è invertibile in $\cO_P(C)$, allora $r \in \cO_P(C)$, altrimenti, se $g_2$ non è ivi invertibile, allora, neanche in $R$ lo 
        è, quindi $r \notin R$, assurdo.
    \end{proof}
    Con "risolvere le singolarità" di una curva proiettiva $C$ si intende trovare una curva proiettiva non-singolare $X$ ed una mappa birazionale $f : X \to C$. Per fare questo si parte dalle curve piane, infatti, se si riesce a dimostrare che per ogni curva piana esiste una tale curva non-singolare, 
    allora, siccome tutte le curve proiettive sono birazionalmente equivalenti ad una curva piana, seguirà che ogni curva è birazionalmente equivalente ad una non-singolare.\\
    L'idea fondamentale è quella dello "scoppiamento dei punti singolari di una curva", che ad un livello intuitivo può essere descritto nel seguente modo: sia $C \subseteq \P^2$ una curva e $P$ un suo punto multiplo. Si rimuove il punto $P$ dal piano e lo si sostituisce con una retta proiettiva $r$. I punti 
    di $r$ corrispondono alle direzioni tangenti a $C$ in $P$. Tutto questo si può fare in modo che il "piano scoppiato", ovvero $B = \P^2 \setminus \{P\} \cup r$ sia ancora una varietà. In tal modo si può costruire una cuva $C' \subseteq B$ birazionalmente equivalente a $C$, ma con singolarità "migliori". \\
    \noindent Studierò prima cosa avviene nel caso affine e poi in quello proiettivo. \\
    Sia $P = (0,0) \in \A^2$ e sia $U = \{(x,y) \in \A^2 : x \neq 0\}$. Considero ora il morfismo $f : U \to \A^1 = k$ definito per ogni $(x,y) \in U$ da $f(x,y) = \frac{y}{x}$. Allora, $G \subseteq \A^1 \times \A^2 = \A^3, G = \{P = (x,y,z) \in \A^3 : y =xz, x \neq 0\}$, è il grafico di $f$. \\
    Sia ora $B = \{P = (x,y,z) : y = xz\}$; siccome $Y - XZ \in k[X,Y,Z]$ è irriducibile, $B$ è varietà. Sia inoltre $\pi : B \to \A^2$, la restrizione a $B$ della proiezione da $\A^3$ sulle prime due coordinate: $\pi$ è un morfismo. \\
    Valgono le seguenti: \begin{itemize}
        \item $\pi(B) = U \cup \{P\}$;
        \item $\pi^{-1}(P) = L = \{(0,0,z): z \in k\}$ e $\pi^{-1}(U) = G$
    \end{itemize}
    Ne segue che $\pi$ è un isomorfismo fra $G$ ed $U$, da cui $G$ è sottovarietà aperta di $B$, e $B$ è la chiusura di $G$ in $\A^3$. Infine $L$ è sottovarietà chiusa di $B$. \\
    Sia $\gvf : \A^2 \to B$ definita per ogni $(x,z) \in \A^2$ da $\gvf(x,z) = (x,xz,z)$: è un isomorfismo con inversa la proiezione sulla prima e terza coordinata. Considero quindi $\psi : \A^2 \to \A^2$, definita come $\psi = \pi \circ \gvf$; è morfismo perché composta di morfismi. \\
    Sia $E = \psi^{-1}(P) = \gvf^{-1}(L) = \{(x,z) \in \A^2 : x = 0\}$. Ne deduco che $\psi : \A^2 \setminus E \to U$ è isomorfismo.\\
    Sia ora $C$ una curva irriducibile del piano affine e sia $C_0 = C \cap U$ una sottovarietà aperta di $C$. Sia $C_0' = \psi^{-1}(C_0)$ e sia $C'$ la chiusura di $C_0'$ in $\A^2$. Sia infine $f : C' \to C$ la restrizione di $\psi$ a $C'$. Siccome $C_0 \subseteq U, f$ è un isomorfismo. 
    Dunque tramite $\tilde{f}$ possiamo identificare $k(C) = k(x,y)$ con $k(C') = k(x,z), y = xz$. \\
    Valgono i seguenti fatti: \begin{itemize}
        \item Sia $C = V(F), F = F_r + \cdots + F_n, F_i$ polinomio omogeneo di grado $i$ in $k[X,Y]$, e siano $r = m_P(C), n = \deg(C)$. Allora, $C' = V(F'), F'(X,Z) = F_r(1,Z) + XF_{r+1}(1,Z) + \cdots + X^{n-r}F_n(1,Z)$.
        \begin{proof}
                $F(X,XZ) = \sum_{i=r}^n X^i F_i(1,Z) = X^r F'(X,Z)$. Ma siccome $F_r(1,Z) \neq 0$, allora, $X$ non divide $F'$. Se per assurdo $F' = GH$, tali che \begin{equation*}
                    F = X^r F'(X,Z) = X^r F' \left( X,\frac{Y}{X} \right) = X^r G \left( X, \frac{Y}{X} \right) H \left( X, \frac{Y}{X} \right)
                \end{equation*}
                ma allora $F$ è riducibile, che è assurdo. Ne segue che anche $F'$ è irriducibile, ne segue che $V(F')$ è un chiuso che contiene $C_0'$. Quindi $C' \subseteq V(F')$. \\
                Viceversa \begin{equation*}
                    F'(P) = 0 \Longrightarrow F(\psi(P)) = 0 \Longrightarrow \psi(P) \in C \Longrightarrow P \in C'
                \end{equation*} 
        \end{proof}
        \item Supposto che la retta $X$ non sia tangente a $C$ in $P$, posso supporre che $F_r(X,Y) = \prod_{i=1}^s (Y-\alpha_iX)^{r_i}$. Allora $f^{-1}(P) = \{P_1,\ldots,P_s\}$, con $P_i = (0,\alpha_i)$ e vale che $m_{P_i}(C') \leq I(P,C \cap E) = r_i$. In particolare, se $P$ è un punto multiplo ordinario, 
        $P_i$ è un punto semplice di $C'$ e $\ord_{P_i}^{C'}(x) = 1$.
        \begin{proof}
            Chiaramente: \begin{equation*}
                f^{-1}(P) = C' \cap E = \{(0,\alpha) : F_r(1,\alpha) = 0\}.
            \end{equation*}
            Inoltre $m_{P_i}(C) \leq I(P_i, F' \cap X) = I(P_i, \prod_{i=1}^r(Z-\alpha_i) \cap X) = r_i$. La parte di enunciato riguardo l'ordine in $P_i$ della funzione $x$ è ovvia.
        \end{proof}
        \item Esiste un intorno affine $W$ di $P$ in $C$ tale che $W' = f^{-1}(W)$ sia una sottovarietà aperta affine di $C'$. Inoltre $\gG(W')$ è un modulo finitamente generato su $\gG(W)$ e $x^{r-1}\gG(W') \subseteq \gG(W)$.
        \begin{proof}
            Sia $F(X,Y) = \sum_{i+ \geq r} a_{ij}X^iY^j$ e sia $H(Y) = \sum_{j \geq r} a_{0j}Y^{j-r}$, ovvero $F(X,Y) = Y^rH(Y) + XG(X,Y)$. Sia infine $h$ l'immagine in $\gG(C)$ di $H$.\\
            Chiaramente $H(0) \neq 0$, quindi $W = C_h = \{Q \in C : h(Q) \neq 0\}$ è un intorno aperto affine di $P$ in $C$. Dimostro che $W = (C_h \cap U) \cup \{P\}$; un'inclusione ($\supseteq$) è ovvia. \\
            Viceversa, \begin{multline*}
                Q \in W \Longrightarrow F(Q) = 0, H(Q) \neq 0 \Longrightarrow x_Q = 0 = y_Q \text{ oppure } \\ x_Q \neq 0 \Longrightarrow Q \in (C_h \cap U) \cup \{P\}
            \end{multline*}
            Infine osservo che \begin{equation*}
                F'(X,Z) = \sum_{i + j \geq r} a_{ij}X^{i+j-r} Z^j = \sum_{i < r} a_{ij}X^{i+j-r}Z^{r-i} + \sum_{i \geq r}a_{ij}X^{i-r}Y^j
            \end{equation*}
            Ma questo prova, valutando in $(x,z)$, che $z^r$ è una combinazione della forma $\sum_{i} b_iz^{r-i}$, da cui  $\gG(W')$ è un modulo finitamente generato su $\gG(W)$. Inoltre per $i \leq r-1$: \begin{equation*}
                x^{r-1}z^i = x^{r-1}\frac{y^i}{x^i} \in \gG(W)
            \end{equation*}
        \end{proof}
    \end{itemize}
    Siano ora $P_1,\ldots,P_t \in \P^2$, e per semplicità nella trattazione suppongo che $P_i \in U_3$ per ogni $i$, quindi $P_i = [a_{i1},a_{i2},1]$. Sia $U = \P^2 \setminus \{P_1,\ldots,P_t\}$. \\
    Definisco i morfismi $f_i : U \to \P^1, f_i(X_1,X_2,X_3) = [X_1 -a_{i1}X_3, X_2 - a_{i2}X_3]$ e sia $f = (f_1,\ldots,f_t) : U \to \P^1 \times \cdots \times \P^1$ la mappa prodotto. Sia infine $G \subseteq U \times \P^1 \times \cdots \times \P^1$ il grafico di $f$.\\
    Fissate quindi le coordinate omogenee $X_1, X_2, X_3$ per $\P^2$ e $Y_{i1}, Y_{i2}$ per la $i$-esima copia di $\P^1$, considero $B = V(Y_{i2} (X_1-a_{i1}X_3) - Y_{i1}(X_2-a_{i2}X_3)$ : $ i \in \{1,\ldots,t\})$.\\
    Chiaramente $G \subseteq B$. Sia $\pi : B \to \P^2$ la restrizione della proiezione e sia, per ogni $i$, $E_i = \pi^{-1}(P_i)$. Valgono i seguenti fatti: \begin{enumerate}
        \item $E_i = \{P_i\} \times \{f_1(P_i)\} \times \cdots \times \P^1 \times \cdots \times \{f_t(P_i)\}$, con $\P^1$ nell'$i$-esima posizione; quindi $E_i$ è isomorfo a $\P^1$.
        \item $B \setminus \cup_{i=1}^t E_i = B \cap (U \times \P^1 \times \cdots \times \P^1) = G$, perciò $\pi$ si restringe ad un isomorfismo tra $B \setminus \cup_{i=1}^t E_i$ e $U$.
        \item Se $T$ è un cambio di coordinate proiettive di $\P^2$, con $T(P_i) = P_i'$, e le mappe $f_i' : \P^2 \setminus \{P_1', \ldots, P_t'\} \to \P^1$ sono definite come le $f_i$, ma con i $P_i'$ al posto dei $P_i$, allora esiste un unico cambio di coordinate 
        proiettive $T_i$ di $\P^1$ tale che $T_i \circ f_i = f_i' \circ T$, dunque $(T_1,\ldots,T_t) \circ f = f' \circ T$. Infine $(T,T_1,\ldots,T_t)$ mappa isomorficamente $G,B,E_i$ nei corrispondenti $G',B',E_i'$ definiti a partire da $f'$.
        \item Se $T_i$ è un cambio di coordinate in $\P^1$ per un $i$ fissato, esiste un cambio di coordinate $T$ di $\P^2$, tale che $T(P_i) = P_i$ e $T_i \circ f_i = f_i \circ T$.
        \item Siano $i \in \{1,\ldots,t\}, Q \in E_i$ fissati; per gli ultimi due punti posso supporre che $P_1 = [0,0,1], Q = [\lambda,1], \exists \lambda \in k$. Sia $\gvf_3^{-1} : \A^2 \to U_3$ il morfismo canonico. Siano $V = U_3 \setminus \{P_j : j \neq i\}, W = (\gvf_3^{-1})^{-1}(V), 
        \psi$ la mappa definita nel caso affine e $W' = \psi^{-1}(W)$. A questo punto considero $\gvf : W' \to \P^2 \times \P^1 \times \cdots \times \P^1$ definita da $$\gvf(x,z) = ((x,xz,1), f_1(x,xz,1), \ldots,(z,1),\ldots,f_t(x,xz,1))$$ con $(z,1)$ in $i$-esima posizione. 
        $\gvf$ è un morfismo, ed è tale che $\pi \circ \gvf = \gvf_3 \circ \psi$. Posto $V' = \gvf(W') = B \setminus (\cup_{j \neq i} E_j \cup V(X_3) \cup V(Y_{i2})), V'$ è intorno aperto di $Q$ in $B$.
        \item $B$ è la chiusura di $G$: se $S$ è un chiuso che contiene $G$, allora $\gvf^{-1}(S)$ è un chiuso di $W'$ che contiene $\gvf^{-1}(G) = W' \setminus V(X)$, che è aperto quindi denso, ne segue che $\gvf^{-1}(S) = W'$, da cui $Q \in S$. Data l'arbitrarietà di $Q$ in 
        $B \setminus G$, segue che $B$ è la chiusura di $G$.
        \item Il morfismo definito su $\P^2 \times \P^1 \times \cdots \times \P^1 \setminus V(X_3Y_{i2})$ verso $\A^2$ che mappa un elemento del suo dominio in $(\frac{x_1}{x_3},\frac{y_{i1}}{y_{i2}})$ è, se ristretto a $V'$, l'inversa di $\gvf$. Allora abbiamo il seguente diagramma 
        commutativo: \\
        \begin{center}
        \begin{tikzcd}
            \A^2 \arrow[r, hookleftarrow] \arrow[d, "\psi"] & W' \arrow[r, "\gvf"] \arrow[d] & V' \arrow[r, hookrightarrow] \arrow[d] & B \arrow[d, "\pi"] \\
            \A^2 \arrow[r, hookleftarrow]                   & W  \arrow[r, "\gvf_3"]         & V  \arrow[r, hookrightarrow]           & \P^2
        \end{tikzcd} 
        \end{center}
        Quindi $\pi$, attorno a $Q$ si comporta analogamente alla mappa $\psi$, dunque valgono per $\pi$ tutte le proprietà di $\psi$.
        \item Sia $C$ una curva irriducibile in $\P^2$. Sia $C_0 = C \cap U, C_0' = \pi^{-1}(C_0) \subseteq G$ e $C'$ la chiusura di $C_0'$ in $B$. Allora $f : C' \to C$ si restringe ad un isomorfismo tra $C_0'$ e $C_0$. Per quanto visto nel punto precedente, tale isomorfismo ha la stessa 
        forma del caso affine.  
    \end{enumerate}
    Vale quindi la seguente: \begin{proposizione}
        Sia $C$ una curva proiettiva piana irriducibile, e suppongo che tutti i suoi punti multipli, siano ordinari. Allora esiste una curva non-singolare $C'$ ed una mappa birazionale da $C'$ a $C$.
    \end{proposizione}
    \begin{proof}
        Si veda \cite{fulton} Capitolo $7$, Paragrafo $3$.
    \end{proof}
    Per quanto visto quindi, se $C$ è una curva piana proiettiva i cui punti multipli sono ordinari, allora $C$ è birazionalmente equivalente ad una curva non singolare. Adesso dimostro che ogni curva piana è birazionalmente equivalente ad una curva i cui punti multipli sono ordinari.\\
    Siano $P = [0,0,1], P' = [0,1,0], P'' = [1,0,0] \in \P^2$; tali punti sono detti fondamentali. Siano $L = V(Z), L' = V(Y), L'' = V(X)$ e queste rette sono dette eccezionali. Sia infine $U = \P^2 \setminus V(XYZ)$.\\
    Definisco $Q : \P^2 \setminus \{P,P',P''\} \to U \cup \{P,P',P''\}, Q(x,y,z) = [yz,xz,xy]$. $Q$ è un morfismo ed è tale che $Q^{-1}(P) = L \setminus \{P',P''\}$ (e simmetricamente per $P',P''$).\\
    Osservo ora che se $[x,y,z] \in U$: \begin{equation*}
        Q^2(x,y,z) = Q(yz,xz,xy) = [xxyz,yxyz,zxyz] = [x,y,z]
    \end{equation*}
    Quindi su $U, Q = Q^{-1}$, quindi $Q$ è un isomorfismo di $U$ con se stesso. In particolare induce una mappa birazionale di $\P^2$ con se stesso. La mappa $Q$ è detta trasformazione quadratica standard.\\
    Sia $C$ una curva irriducibile in $\P^2$, e suppongo non sia una retta eccezionale. Allora $C \cap U$ è una curva chiusa in $U$. Sia $C'$ la chiusura di $Q(C \cap U) = Q^{-1}(C \cap U)$ in $\P^2$; $Q$ si restringe ad un morfismo tra $C' \setminus \{P,P',P''\}$ e $C$. Inoltre $(C')' = C$ 
    perché $Q^2 = \text{id}_U$.\\
    Sia $F \in k[X,Y,Z]$, tale che $C = V(F), n = \deg(F)$ definisco la trasformata algebrica di $F$ come: $F^Q = F(YZ,XZ,XY)$. Tale polinomio è omogeneo di grado $2n$.\\
    Valgono i seguenti fatti: \begin{enumerate}
        \item Se $m_P(C) = r$, allora, $Z^r$ è la più alta potenza di $Z$ che divide $F^Q$, e simmetricamente in $P',P''$. \\
    \end{enumerate}
    Se $F^Q = X^{r''}Y^{r'}Z^rF'$, il polinomio omogeneo $F'$ è detto trasformata propria di $F$. \begin{enumerate} \setcounter{enumi}{1}
        \item $\deg(F') = 2n - r - r' - r'', (F')' = F, V(F') = C'$.
        \item $m_P(C') = n - r' - r''$, e simmetricamente per $P',P''$.
    \end{enumerate}
    Suppongo ora che nessuna retta eccezionale sia tangente a $C$ in un punto fondamentale. Una tale curva si dice essere in buona posizione.
    \begin{enumerate}\setcounter{enumi}{3}
        \item Se $C$ è in buona posizione, allora, anche $C'$ lo è. \\
    \end{enumerate}
    Sia $C$ in buona posizione e che $P \in C$; sia $C_0 = (C \cap U) \cup \{P\}, C_0' = C' \setminus V(XY)$. Allora $f : C_0' \to C_0$ è la restrizione di $Q$
    Considero ora il polinomio affinizzato $F_* = F(X,Y,1)$, e la curva affinizzata $C_* = V(F_*) \subseteq \A^2$; definisco $(F_*)' = F(X,XZ,1)X^{-r} C_*' = V(F_*') \subseteq \A^2$ ed $f_* : C_*' \to C_*, f_*(x,z) = (x,xz)$.
    \begin{enumerate} \setcounter{enumi}{4}
        \item Esiste un intorno $W$ di $(0,0)$ in $C_*$ e isomorfismi $\gvf : W \to C_0, \gvf' : W' = f_*^{-1}(W) \to C_0'$ tali che $\gvf \circ f_* = \gvf' \circ f$.
        \item Se $C$ è in buona posizione e $P_1,\ldots,P_s$ sono punti non-fondamentali su $C' \cap L$, allora, $m_{P_i}(C') \leq I(P_i,C' \cap L)$. \\
    \end{enumerate}
    Si dice che la curva $C$ è in posizione eccellente se interseca $L$ trasversalmente in $n$ punti non-fondamentali distinti, ed interseca trasversalmente $L',L''$ ciascuna in $n-r$ punti non-fondamentali distinti.
    \begin{enumerate} \setcounter{enumi}{6}
        \item Se $C$ è in posizione eccellente, allora gli unici punti multipli di $C'$ sono quelli in $C' \cap U$ che corrispondono a quelli in $C \cap U$ e questa corrispondenza rispetta la molteplicità dei punti e se questi sono ordinari o meno; $P,P',P''$ che sono ordinari di molteplicità 
        $n,n-r,n-r$ rispettivamente e dei punti $P_1,\ldots,P_s$ non fondamentali su $C' \cap L$, di molteplicità tali che $m_{P_i}(C') \leq I(P_i,C'\cap L), \sum_{i=1}^s I(P_i,C'\cap L) = r$.\\
    \end{enumerate}
    Per $C$ curva piana proiettiva irriducibile, di grado $n$ si definisce $g_*(C) = \frac{(n-1)(n-2)}{2} - \sum_{P \in C}\frac{r_P(r_P-1)}{2}$, dove $r_P = m_P(C)$. \begin{enumerate} \setcounter{enumi}{7}
        \item Se $C$ è in posizione eccellente allora, $g_*(C') = g_*(C)-\sum_{i=1}^s\frac{r_i(r_i-1)}{2}$, dove $r_i = m_{P_i}(C')$ e $P_1,\ldots,P_s$ sono i punti non fondamentali di $C' \cap L$.
    \end{enumerate}
    Abbandono ora le notazioni per punti fondamentali e rette eccezionali che ho usato fino ad ora.
    \begin{lemma}
        Sia $F$ una curva piana proiettiva irriducibile e $P$ un suo punto, allora esiste un cambio di coordinate $T$, tale che $F^T$ è in posizione eccellente e $T(0,0,1) = P$.
    \end{lemma}
    \begin{proof}
        Si veda \cite{fulton} Capitolo $7$, Paragrafo $4$.
    \end{proof}
    Se $T$ è un cambio di coordinate omogenee, allora, $Q \circ T$ è detta trasforazione quadratica e $(F^T)'$ è detto trasformata quadratica di $F$. Se $F^T$ è in posizione eccellente e $T(0,0,1) = P$, allora si dice che la trasformata è centrata in $P$.\\
    Se $F = F_1,\ldots,F_n = G$ sono curve e $F_i$ è trasformata quadratica di $F_{i-1}$, allora, si dice che $F$ è trasformata in $G$ da una sequenza finita di trasformazioni quadratiche.
    \begin{proposizione}
        Tramite un numero finito di trasformazioni quadratiche, ogni curva piana proiettiva irriducibile può essere trasformata in una curva i cui punti multipli sono ordinari.
    \end{proposizione}
    \begin{proof}
        Si veda \cite{fulton} Capitolo $7$, Paragrafo $4$.
    \end{proof}
    \begin{teorema}\label{teo:model}
        Sia $C$ una curva proiettiva. Allora esiste una curva proiettiva non-singolare $X$ ed una mappa birazionale $f$ da $X$ a $C$. Se $f': X' \to C$ sono un'altra mappa birazionale ed un altra curva non-singolare, allora esiste un unico isomorfismo $g: X \to X'$ tale che $f' \circ g = f$.
    \end{teorema}
    \begin{proof}
        Si veda \cite{fulton} Capitolo $7$, Paragrafo $5$.
    \end{proof}
    \begin{corollario}
        Esiste un corrispondenza biiettiva fra curve proiettive non-singolari e campi di funzioni in una variabile. Se $X,X'$ sono due tali curve, i morfismi dominanti da $X$ a $X'$ corrispondono agli omomorfismi da $k(X')$ a $k(X)$.
    \end{corollario}
    Sia $C$ una curva proiettiva. $f : X \to C$ come nel Teorema \ref{teo:model}. Si dice che $X$ è il modello non-singolare di $C$ o di $K = k(C)$. Si identifica $k(X)$ con $k(C)$ tramite $\tilde{f}$. I punti di $X$ sono detti posti di $C$ ed un posto $Q \in X$ si dice centrato in $P \in C$ se 
    $f(Q) = P$.\\
    Suppongo ora che $C$ sia piana, $Q \in X, f(Q) = P \in C$. Per ogni altra curva piana $G$, eventualmente riducibile, sia $G_* \in \cO_P(\P^2)$ e sia $g$ l'immagine di $G_*$ in $\cO_P(C) \subseteq k(C) = k(X)$. Definisco $\ord_Q(G) = \ord_Q(g)$.\\
    \begin{proposizione}\label{prop:molt-int}
        Sia $C$ una curva piana proiettiva irriducibile, $P \in C, f : X \to C$ come sopra. Sia $G$ un'altra curva piana, eventualmente riducibile. Allora: $I(P,C \cap G) = \sum_{Q \in f^{-1}(P)}\ord_Q(G)$.
    \end{proposizione}
    \begin{proof}
        Si veda \cite{fulton} Capitolo $7$, Paragrafo $5$.
    \end{proof}
    \begin{lemma}
        Se $P$ è un punto multiplo ordinario su $C$ di molteplicità $r$ e sia $f^{-1}(P) = \{P_1,\ldots,P_r\}$. Se $z \in k(C)$ e $\ord_{P_i}(z) \geq r-1$, allora $z \in \cO_P(C)$.
    \end{lemma}
    \begin{proof}
        Si veda \cite{fulton} Capitolo $7$, Paragrafo $5$.
    \end{proof}
    \begin{proposizione}\label{prop:n-c}
        Siano $F$ una curva piana proiettiva irriducibile e $P$ un suo punto multiplo ordinario di molteplicità $r$ e siano $P_1,\ldots,P_r$ i posti di $F$ centrati in $P$. Siano inoltre $G,H$ altre due curve piane, eventualmente riducibili. Allora le condizioni di Noether sono soddisfatte in $P$ rispetto a 
        $F,G,H$ se e solo se $\forall i \in \{1,\ldots,r\} \ord_{P_i}(H) \geq \ord_{P_i}(G) + r-1$.
    \end{proposizione}
    \begin{proof}
        Si veda \cite{fulton} Capitolo $7$, Paragrafo $5$.
    \end{proof}



































    









   